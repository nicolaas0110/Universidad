\documentclass[11pt]{article}
\usepackage{indentfirst} 		 % First line paragraph indentation
\usepackage{etoolbox} 			% Used for being able to utilise if statements in the language detection.
%\setlength{\parskip}{\baselineskip}	% With this line, it is not needed the use of \\ at the end of each paragraph for...
					% ...spacing purposes, but the spacing can be extrange sometimes


% ========================= VARIABLES TO MODIFY =================================

\def\LANGUAGE{ES} %EN, ES		%Language in which the document is going to be written.

\def\reportTile{Apuntes}			%Document's title
\def\subject{EDO}			%Subject

\def\writter{Nicolas Muñoz}		%Author(s)


\def\leftUpperHeader{\subject}		%\subject
\def\rightUpperHeader{\leftmark}	%\leftmark for current section


% Do NOT modify anything from here until the beginning of the document.
%============================== DATA PROCESSING ============================
\ifdefstring{\LANGUAGE}{EN}
{

\def\course{Licenciatura en Matemática}
\def\university{Pontificia Universidad Católica - Chile}
\def\pageCounterName{PAGE} 		% In the footer will be shown this... 
\def\pageSeparator{OUT OF} 		% ...text in addition of the page number.

}
{
\usepackage[utf8]{inputenc}		%Tildes en caso de no usar arch
\usepackage[spanish, es-tabla]{babel} 	%La opción es-tabla, hace que por defecto las tablas se llamen "Tabla", en vez de "Cuadro".
\def\course{Licenciatura en Matemática}
\def\university{Pontificia Universidad Católica - Chile}
\def\pageCounterName{PÁGINA}		%En el pie de página aparece este contenido más el número de página
\def\pageSeparator{DE} 			%DE, OUT OF

}


%===================== USEFUL VARIABLES  =============================
\def\imageSize{0.6}

%===================== PACKAGES TO USE  =============================
\usepackage{amsmath} 			% Allows me the use of matrix in math mode
\setcounter{MaxMatrixCols}{20} 		% Increases the maximum number of columns in matrix from 10 to 20.
\usepackage{pdfpages} 			% Attach pdfs using \includepdf[pages=initial-final]{path.pdf}
\usepackage{lastpage}			% Used to reference the last page and include it in the footer.
\usepackage{xcolor}			% Change color of text and so on. \textcolor{color}{text} is the one I use the most.
\usepackage{placeins} 			% Allows the use of the command \FloatBarrier
\usepackage{setspace}			% Allows the use of the spacing environment to change the space between lines.
\usepackage{graphicx}			% Figures addition
\usepackage{geometry}			% Changes the geometry of the pages
\usepackage[small, bf]{caption}		% Decreases the size of captions and turns them bold text.
\usepackage{subcaption}			% Include subfigures.
\usepackage{hyperref}			% Allows clickable references.
\hypersetup{colorlinks=true, allcolors=black} 	% Colors of links
\usepackage{pdflscape}			% Allows the use of the environment lscape for landscape pages

\usepackage{fancyhdr}			% Modify header and footer
\usepackage{bm}				% Allows to use bold text in math mode with \bm
\usepackage[makeroom]{cancel}		% Allows the use of \cancelto{}{} to cross out equations
\usepackage{titlesec}			% Change title format


%========== CHANGE TITLE FORMAT  ======================
%\titleformat{\section}
%{\bfseries}	% format
%{\thesection}	% label
%{0.3cm}		% separation between label and body
%{}		% code preceding title body
%[]		% code following title body


%========== HEADER AND FOOTER CONFIGURATION ======================
\pagestyle{fancy}
% HEADER[EVEN PAGES]{ODD PAGES}
% Left header
\lhead[
	\scriptsize{\MakeUppercase{\leftUpperHeader}} % Even pages
]
{
	\scriptsize{\MakeUppercase{\leftUpperHeader}} % Odd pages
}

% Central header
\chead[]{}

% Right header
\rhead[
\scriptsize{\rightUpperHeader} % Even pages
]
{
\scriptsize{\rightUpperHeader} % Odd pages
}

\renewcommand{\headrulewidth}{0.8pt}	% Width of the header rule

% FOOTER[EVEN PAGES]{ODD PAGES}
% Left footer
\lfoot[]{}

% Central footer
\cfoot[
\tiny{\pageCounterName\space\thepage\space \pageSeparator\space\pageref{LastPage}} % Even pages
]
{
\tiny{\pageCounterName\space\thepage\space \pageSeparator\space\pageref{LastPage}} % Odd pages
}

% Right footer
\rfoot[]{}
\renewcommand{\footrulewidth}{0.8pt} %Width of the footer rule


%================================== USER OWN FUNCTIONS ===============================================
\newcommand{\PDEA}[1]{\cdot 10^{#1}} % Función para escribir más rápidamente multiplicaciones por potencias de 10. PDEA = Por Diez Elevado A

\usepackage{xargs}	%Permite manejar argumentos opcionales en los comandos creados por el usuario

%Incluir imágenes, la sintaxis es la siguiente:
%\IncludeImage{ruta}[escalado][pie figura][etiqueta], los corchetes son opcionales
\newcommandx*{\IncludeImage}[4][2=1, 3=, 4=]{
	%#1 es la ruta a la imagen
	%#2 es el escalado
	%#3 es el pie de figura
	%#4 es la etiqueta
	
	\begin{figure}[h!]
		\centering
		\includegraphics[width=#2\linewidth]{#1}
		%Si se especifica pie de figura...
		\ifblank{#3}{}{
			\caption{#3}	
		}
		%Si se especifica etiqueta...
		\ifblank{#4}{}{
			\label{#4}	
		}
	\end{figure}
	\FloatBarrier
}



%================================ DOCUMENT BEGINNING  =============================================
\begin{document}
%***************** TITLE PAGE (DO NOT MODIFY) ******************************
\begin{titlepage}
	\newgeometry{top=2cm, bottom=3.5cm}

  %Logos Escuela y universidad
	\def\logoSize{0.2}
	\begin{figure}
  \hfill
  \includegraphics[width=\logoSize\linewidth]{Image/UC COLOR-01.png}
	\end{figure}

	%Espacio entre logos y título
	\vspace*{1.75cm}

	%Título con interlineado aumentado
	\begin{spacing}{2}
	\centering{
	\Huge{
		\textbf{\reportTile}
	}
	}
	\end{spacing}
	
	%Línea horizontal de la portada
	\hrule
	
	%Se baja al final de la hoja
	\vfill
	\begin{flushright}
	\LARGE{\writter}

	\vspace{2cm}
	
	\Large
	\subject\\
	\course\\
	\university
	
	\vspace{1cm}
	
	\today
	\end{flushright}
	
\end{titlepage}
	
\newgeometry{top=3cm, bottom=3cm}

\tableofcontents
\cleardoublepage

\section{Introducción a las EDO´s}
\label{sec:intro-edo}

\subsection{Generalidades y Definiciones}
\begin{definition}
Una **ecuación diferencial ordinaria (EDO)** de orden $k \in \mathbb{N}$ es una relación funcional entre la variable real $t \in I$ (donde $I \subset \mathbb{R}$ es un intervalo abierto), una función $y: I \rightarrow \mathbb{R}^m$, y sus derivadas $y', y'', \dots, y^{(k)}$. Esta relación se expresa a través de la fórmula:
$$F(t, y(t), y'(t), \dots, y^{(k)}(t)) = 0, \quad \forall t \in I \quad (*)$$
donde $F: J \times \mathbb{R}^m \times \dots \times \mathbb{R}^m \rightarrow \mathbb{R}^n$ es una función dada.
\end{definition}

\begin{definition}
Una **solución** de la EDO $(*)$ es una función $\phi \in C^k(I; \mathbb{R}^m)$ tal que $F(t, \phi(t), \phi'(t), \dots, \phi^{(k)}(t)) = 0$ para todo $t \in I$.
\end{definition}

Se asume que la EDO se puede "resolver" con respecto a la derivada de mayor orden $y^{(k)}$, resultando en la forma canónica:
$$y^{(k)}(t) = f(t, y(t), y'(t), \dots, y^{(k-1)}(t)), \quad \forall t \in I$$
donde $f \in C(I \times \mathbb{R}^m \times \dots \times \mathbb{R}^m; \mathbb{R}^m)$. Esto es una hipótesis razonable por el Teorema de la Función Implícita, asumiendo que $\frac{\partial F}{\partial y^{(k)}} \ne 0$.

\begin{definition}
Una EDO en su forma canónica se dice **lineal** cuando tiene la forma:
$$y^{(k)}(t) = \sum_{j=0}^{k-1} a_j(t) y^{(j)}(t) + g(t), \quad \forall t \in I$$
donde $a_j \in C(I, M_{m \times m}(\mathbb{R}))$ y $g \in C(I; \mathbb{R}^m)$ son funciones dadas.
\end{definition}

\begin{remark}
Un sistema lineal es:
\begin{itemize}
    \item **Homogéneo** si $g(t) \equiv 0$.
    \item **Autónomo** si $f$ no depende de $t \in I$.
\end{itemize}
\end{remark}

\subsection{Curiosidades y Reducción de Orden}
\label{sec:curiosidades}

\begin{enumerate}
    \item Las funciones $\phi_1(t) = e^{-2t}$ y $\phi_2(t) = e^{-3t}$ son soluciones de la EDO $y''(t) + 5y'(t) + 6y(t) = 0$. Por el principio de superposición, cualquier combinación lineal $c_1\phi_1(t) + c_2\phi_2(t)$ también es una solución, lo que implica infinitas soluciones.
    \item La única solución real de la EDO $(y(t))^2 + (y'(t))^2 = 0$ para todo $t \in \mathbb{R}$ es la función idénticamente nula.
    \item **Invariancia por traslación:** Si $\phi$ es una solución de un sistema autónomo, entonces $\overline{\phi}(t) = \phi(t-t_0)$ también es una solución para cualquier constante $t_0$.
    \item **Reducción a un sistema autónomo de primer orden:** Cualquier sistema de EDOs se puede reducir a un sistema autónomo del primer orden. Para un sistema de orden $k \ge 2$, se definen nuevas variables $u_0=y, u_1=y', \dots, u_{k-1}=y^{(k-1)}$. Esto lleva a un sistema de primer orden. Si se introduce una variable adicional $u_k = t$, se puede obtener un sistema de primer orden autónomo.
\end{enumerate}

\section{Problemas de Cauchy}
\label{sec:cauchy}

\begin{definition}[Problema de Cauchy (PC)]
Un **problema de Cauchy** para un sistema de EDOs de primer orden es un problema de valor inicial que se expresa como:
$$(PC) \begin{cases} y'(t) = f(t, y(t)) & \forall t \in I \\ y(t_0) = y_0 \end{cases}$$
donde $t_0 \in I$ es el punto inicial y $y_0 \in \mathbb{R}^m$ es el valor inicial. Para que la solución sea única, se necesita agregar una condición inicial.
\end{definition}

Para EDOs de orden superior, el problema de Cauchy incluye condiciones iniciales para la función y sus primeras $k-1$ derivadas.
$$\begin{cases} y^{(k)}(t) = f(t, y(t), y'(t), \dots, y^{(k-1)}(t)) & \forall t \in I \\ y(t_0) = y_0, y'(t_0) = y_1, \dots, y^{(k-1)}(t_0) = y_{k-1} \end{cases}$$

\section{Resolución para EDOs de Primer Orden}
\label{sec:resolucion-explícita}

\subsection{Variables Separables}
Una EDO de variables separables tiene la forma $y' = g(t)/h(y)$, donde $g$ y $h$ son funciones continuas y $h(y) \ne 0$. La solución se encuentra al separar las variables e integrar:
$$h(\phi(t))\phi'(t) = g(t) \Rightarrow \int h(y)dy = \int g(t)dt + C$$
Esto da una representación implícita de la solución.

\subsection{EDOs Lineales de Primer Orden}
Una EDO lineal de primer orden tiene la forma $y'(t) + p(t)y(t) = f(t)$. El método de resolución implica dos pasos:

\begin{enumerate}
    \item **Resolver la ecuación homogénea asociada:** $y'(t) + p(t)y(t) = 0$. La solución homogénea es $y_h(t) = Ce^{-\int p(t)dt}$.
    \item **Encontrar una solución particular:** Una solución particular $y_p(t)$ se encuentra multiplicando la ecuación no-homogénea por el factor integrante $e^{\int p(t)dt}$ y luego integrando. La solución general es la suma de las soluciones homogénea y particular: $y(t) = y_h(t) + y_p(t)$.
\end{enumerate}

\section{Soluciones de EDOs Autónomas de Primer Orden}
\label{sec:soluciones-autonomas}
Para una EDO autónoma de primer orden de la forma $y'(t) = f(y(t))$ con una condición inicial $y(0) = y_0$. Si $f(y_0) \ne 0$, se puede encontrar una solución única al integrar la ecuación $\frac{y'(t)}{f(y(t))} = 1$. Esto lleva a la expresión $\int_{y_0}^{y(t)} \frac{du}{f(u)} = t$. Si $\psi(y) = \int_{y_0}^{y} \frac{du}{f(u)}$, la solución es $\phi(t) = \psi^{-1}(t)$.
\begin{proposition}
Si $f: I \rightarrow \mathbb{R}$ es continua y $y_0 \in I$ tal que $f(y_0) \ne 0$, la solución al problema de valor inicial es monótona en su intervalo de definición.
\end{proposition}

\subsection{Intervalo Maximal de Existencia}
El intervalo de existencia de la solución puede ser finito. Por ejemplo, en el caso de $y' = y^2$ con $y(0) = y_0 > 0$, la solución es $\phi(t) = \frac{y_0}{1-y_0t}$, que solo existe en el intervalo $(-\infty, 1/y_0)$ y "explota" en $t = 1/y_0$. Este es un ejemplo de una **solución maximal**.

Es importante notar que las soluciones pueden no ser únicas si $f(y_0)=0$ en la condición inicial.

\end{document}
