\documentclass[11pt]{article}
\usepackage{indentfirst} 		 % First line paragraph indentation
\usepackage{etoolbox} 			% Used for being able to utilise if statements in the language detection.
%\setlength{\parskip}{\baselineskip}	% With this line, it is not needed the use of \\ at the end of each paragraph for...
					% ...spacing purposes, but the spacing can be extrange sometimes


% ========================= VARIABLES TO MODIFY =================================

\def\LANGUAGE{ES} %EN, ES		%Language in which the document is going to be written.

\def\reportTile{Apuntes}			%Document's title
\def\subject{EDO}			%Subject

\def\writter{Nicolas Muñoz}		%Author(s)


\def\leftUpperHeader{\subject}		%\subject
\def\rightUpperHeader{\leftmark}	%\leftmark for current section


% Do NOT modify anything from here until the beginning of the document.
%============================== DATA PROCESSING ============================
\ifdefstring{\LANGUAGE}{EN}
{

\def\course{Licenciatura en Matemática}
\def\university{Pontificia Universidad Católica - Chile}
\def\pageCounterName{PAGE} 		% In the footer will be shown this... 
\def\pageSeparator{OUT OF} 		% ...text in addition of the page number.

}
{
\usepackage[utf8]{inputenc}		%Tildes en caso de no usar arch
\usepackage[spanish, es-tabla]{babel} 	%La opción es-tabla, hace que por defecto las tablas se llamen "Tabla", en vez de "Cuadro".
\def\course{Licenciatura en Matemática}
\def\university{Pontificia Universidad Católica - Chile}
\def\pageCounterName{PÁGINA}		%En el pie de página aparece este contenido más el número de página
\def\pageSeparator{DE} 			%DE, OUT OF

}


%===================== USEFUL VARIABLES  =============================
\def\imageSize{0.6}

%===================== PACKAGES TO USE  =============================
\usepackage{amsmath} 			% Allows me the use of matrix in math mode
\setcounter{MaxMatrixCols}{20} 		% Increases the maximum number of columns in matrix from 10 to 20.
\usepackage{pdfpages} 			% Attach pdfs using \includepdf[pages=initial-final]{path.pdf}
\usepackage{lastpage}			% Used to reference the last page and include it in the footer.
\usepackage{xcolor}			% Change color of text and so on. \textcolor{color}{text} is the one I use the most.
\usepackage{placeins} 			% Allows the use of the command \FloatBarrier
\usepackage{setspace}			% Allows the use of the spacing environment to change the space between lines.
\usepackage{graphicx}			% Figures addition
\usepackage{geometry}			% Changes the geometry of the pages
\usepackage[small, bf]{caption}		% Decreases the size of captions and turns them bold text.
\usepackage{subcaption}			% Include subfigures.
\usepackage{hyperref}			% Allows clickable references.
\hypersetup{colorlinks=true, allcolors=black} 	% Colors of links
\usepackage{pdflscape}			% Allows the use of the environment lscape for landscape
\usepackage{amssymb}
\usepackage{amsthm}
\usepackage{fancyhdr}			% Modify header and footer
\usepackage{bm}				% Allows to use bold text in math mode with \bm
\usepackage[makeroom]{cancel}		% Allows the use of \cancelto{}{} to cross out equations
\usepackage{titlesec}
\usepackage{mathtools}



\theoremstyle{definition} % hace que el cuerpo sea upright (no itálico)
\newtheorem{theorem}{Teorema}[section]
\newtheorem{lemma}[theorem]{Lema}
\newtheorem{proposition}[theorem]{Proposición}
\newtheorem{corollary}[theorem]{Corolario}
\newtheorem{definition}[theorem]{Definición}
\newtheorem{example}[theorem]{Ejemplo}
\newtheorem{remark}[theorem]{Observación}
\newtheorem{note}[theorem]{Nota}
\newcommand{\R}{\mathbb{R}}
\newcommand{\N}{\mathbb{N}}
\newcommand{\Q}{\mathbb{Q}}
\newcommand{\Z}{\mathbb{Z}}

 % Change title format


%========== CHANGE TITLE FORMAT  ======================
%\titleformat{\section}
%{\bfseries}	% format
%{\thesection}	% label
%{0.3cm}		% separation between label and body
%{}		% code preceding title body
%[]		% code following title body


%========== HEADER AND FOOTER CONFIGURATION ======================
\pagestyle{fancy}
% HEADER[EVEN PAGES]{ODD PAGES}
% Left header
\lhead[
	\scriptsize{\MakeUppercase{\leftUpperHeader}} % Even pages
]
{
	\scriptsize{\MakeUppercase{\leftUpperHeader}} % Odd pages
}

% Central header
\chead[]{}

% Right header
\rhead[
\scriptsize{\rightUpperHeader} % Even pages
]
{
\scriptsize{\rightUpperHeader} % Odd pages
}

\renewcommand{\headrulewidth}{0.8pt}	% Width of the header rule

% FOOTER[EVEN PAGES]{ODD PAGES}
% Left footer
\lfoot[]{}

% Central footer
\cfoot[
\tiny{\pageCounterName\space\thepage\space \pageSeparator\space\pageref{LastPage}} % Even pages
]
{
\tiny{\pageCounterName\space\thepage\space \pageSeparator\space\pageref{LastPage}} % Odd pages
}

% Right footer
\rfoot[]{}
\renewcommand{\footrulewidth}{0.8pt} %Width of the footer rule


%================================== USER OWN FUNCTIONS ===============================================
\newcommand{\PDEA}[1]{\cdot 10^{#1}} % Función para escribir más rápidamente multiplicaciones por potencias de 10. PDEA = Por Diez Elevado A

\usepackage{xargs}	%Permite manejar argumentos opcionales en los comandos creados por el usuario

%Incluir imágenes, la sintaxis es la siguiente:
%\IncludeImage{ruta}[escalado][pie figura][etiqueta], los corchetes son opcionales
\newcommandx*{\IncludeImage}[4][2=1, 3=, 4=]{
	%#1 es la ruta a la imagen
	%#2 es el escalado
	%#3 es el pie de figura
	%#4 es la etiqueta
	
	\begin{figure}[h!]
		\centering
		\includegraphics[width=#2\linewidth]{#1}
		%Si se especifica pie de figura...
		\ifblank{#3}{}{
			\caption{#3}	
		}
		%Si se especifica etiqueta...
		\ifblank{#4}{}{
			\label{#4}	
		}
	\end{figure}
	\FloatBarrier
}



%================================ DOCUMENT BEGINNING  =============================================
\begin{document}
%***************** TITLE PAGE (DO NOT MODIFY) ******************************
\begin{titlepage}
	\newgeometry{top=2cm, bottom=3.5cm}

  %Logos Escuela y universidad
	\def\logoSize{0.2}
	\begin{figure}
  \hfill
  \includegraphics[width=\logoSize\linewidth]{Image/UC COLOR-01.png}
	\end{figure}

	%Espacio entre logos y título
	\vspace*{1.75cm}

	%Título con interlineado aumentado
	\begin{spacing}{2}
	\centering{
	\Huge{
		\textbf{\reportTile}
	}
	}
	\end{spacing}
	
	%Línea horizontal de la portada
	\hrule
	
	%Se baja al final de la hoja
	\vfill
	\begin{flushright}
	\LARGE{\writter}

	\vspace{2cm}
	
	\Large
	\subject\\
	\course\\
	\university
	
	\vspace{1cm}
	
	\today
	\end{flushright}
	
\end{titlepage}
	
\newgeometry{top=3cm, bottom=3cm}

\tableofcontents
\cleardoublepage

\section{Introducción a las EDO´s}
\label{sec:intro-edo}

\subsection{Generalidades y Definiciones}
\begin{definition}
Una **ecuación diferencial ordinaria (EDO)** de orden $k \in \mathbb{N}$ es una relación funcional entre la variable real $t \in I$ (donde $I \subset \mathbb{R}$ es un intervalo abierto), una función $y: I \rightarrow \mathbb{R}^m$, y sus derivadas $y', y'', \dots, y^{(k)}$. Esta relación se expresa a través de la fórmula:
$$F(t, y(t), y'(t), \dots, y^{(k)}(t)) = 0, \quad \forall t \in I \quad (*)$$
donde $F: J \times \mathbb{R}^m \times \dots \times \mathbb{R}^m \rightarrow \mathbb{R}^n$ es una función dada.
\end{definition}

\begin{definition}
Una **solución** de la EDO $(*)$ es una función $\phi \in C^k(I; \mathbb{R}^m)$ tal que $F(t, \phi(t), \phi'(t), \dots, \phi^{(k)}(t)) = 0$ para todo $t \in I$.
\end{definition}

Se asume que la EDO se puede resolver con respecto a la derivada de mayor orden $y^{(k)}$, resultando en la forma canónica:
$$y^{(k)}(t) = f(t, y(t), y'(t), \dots, y^{(k-1)}(t)), \quad \forall t \in I$$
donde $f \in C(I \times \mathbb{R}^m \times \dots \times \mathbb{R}^m; \mathbb{R}^m)$. Esto es una hipótesis razonable por el Teorema de la Función Implícita, asumiendo que $\frac{\partial F}{\partial y^{(k)}} \ne 0$.

\begin{definition}
Una EDO en su forma canónica se dice **lineal** cuando tiene la forma:
$$y^{(k)}(t) = \sum_{j=0}^{k-1} a_j(t) y^{(j)}(t) + g(t), \quad \forall t \in I$$
donde $a_j \in C(I, M_{m \times m}(\mathbb{R}))$ y $g \in C(I; \mathbb{R}^m)$ son funciones dadas.
\end{definition}

\begin{remark}
Un sistema lineal es:
\begin{itemize}
    \item **Homogéneo** si $g(t) \equiv 0$.
    \item **Autónomo** si $f$ no depende de $t \in I$.
\end{itemize}
\end{remark}

\subsection{Curiosidades y Reducción de Orden}
\label{sec:curiosidades}

\begin{enumerate}
    \item Las funciones $\phi_1(t) = e^{-2t}$ y $\phi_2(t) = e^{-3t}$ son soluciones de la EDO $y''(t) + 5y'(t) + 6y(t) = 0$. Por el principio de superposición, cualquier combinación lineal $c_1\phi_1(t) + c_2\phi_2(t)$ también es una solución, lo que implica infinitas soluciones.
    \item La única solución real de la EDO $(y(t))^2 + (y'(t))^2 = 0$ para todo $t \in \mathbb{R}$ es la función idénticamente nula.
    \item **Invariancia por traslación:** Si $\phi$ es una solución de un sistema autónomo, entonces $\overline{\phi}(t) = \phi(t-t_0)$ también es una solución para cualquier constante $t_0$.
    \item **Reducción a un sistema autónomo de primer orden:** Cualquier sistema de EDOs se puede reducir a un sistema autónomo del primer orden. Para un sistema de orden $k \ge 2$, se definen nuevas variables $u_0=y, u_1=y', \dots, u_{k-1}=y^{(k-1)}$. Esto lleva a un sistema de primer orden. Si se introduce una variable adicional $u_k = t$, se puede obtener un sistema de primer orden autónomo.
\end{enumerate}
\section{Problemas de Cauchy}
\label{sec:cauchy}

\begin{definition}[Problema de Cauchy (PC)]
Un **problema de Cauchy** para un sistema de EDOs de primer orden es un problema de valor inicial que se expresa como:
$$(PC) \begin{cases} y'(t) = f(t, y(t)) & \forall t \in I \\ y(t_0) = y_0 \end{cases}$$
donde $t_0 \in I$ es el punto inicial y $y_0 \in \mathbb{R}^m$ es el valor inicial. Para que la solución sea única, se necesita agregar una condición inicial.
\end{definition}

Para EDOs de orden superior, el problema de Cauchy incluye condiciones iniciales para la función y sus primeras $k-1$ derivadas.

\[
\begin{cases} y^{(k)}(t) = f(t, y(t), y'(t), \dots, y^{(k-1)}(t)) & \forall t \in I \\ y(t_0) = y_0, y'(t_0) = y_1, \dots, y^{(k-1)}(t_0) = y_{k-1} \end{cases}
\]

\section{Resolución para EDOs de Primer Orden}
\label{sec:resolucion-explícita}

\subsection{Variables Separables}
Una EDO de variables separables tiene la forma $y' = g(t)/h(y)$, donde $g$ y $h$ son funciones continuas y $h(y) \ne 0$. La solución se encuentra al separar las variables e integrar:

\[
  h(\phi(t))\phi'(t) = g(t) \Rightarrow \int h(y)dy = \int g(t)dt + C
\]

Esto da una representación implícita de la solución.

\subsection{EDOs Lineales de Primer Orden}
Una EDO lineal de primer orden tiene la forma $y'(t) + p(t)y(t) = f(t)$. El método de resolución implica dos pasos:

\begin{enumerate}
    \item **Resolver la ecuación homogénea asociada:** $y'(t) + p(t)y(t) = 0$. La solución homogénea es $y_h(t) = Ce^{-\int p(t)dt}$.
    \item **Encontrar una solución particular:** Una solución particular $y_p(t)$ se encuentra multiplicando la ecuación no-homogénea por el factor integrante $e^{\int p(t)dt}$ y luego integrando. La solución general es la suma de las soluciones homogénea y particular: $y(t) = y_h(t) + y_p(t)$.
\end{enumerate}

\section{Soluciones de EDOs Autónomas de Primer Orden}
\label{sec:soluciones-autonomas}
Para una EDO autónoma de primer orden de la forma $y'(t) = f(y(t))$ con una condición inicial $y(0) = y_0$. Si $f(y_0) \ne 0$, se puede encontrar una solución única al integrar la ecuación $\frac{y'(t)}{f(y(t))} = 1$. Esto lleva a la expresión $\int_{y_0}^{y(t)} \frac{du}{f(u)} = t$. Si $\psi(y) = \int_{y_0}^{y} \frac{du}{f(u)}$, la solución es $\phi(t) = \psi^{-1}(t)$.
\begin{proposition}
Si $f: I \rightarrow \mathbb{R}$ es continua y $y_0 \in I$ tal que $f(y_0) \ne 0$, la solución al problema de valor inicial es monótona en su intervalo de definición.
\end{proposition}

\subsection{Intervalo Maximal de Existencia}
El intervalo de existencia de la solución puede ser finito. Por ejemplo, en el caso de $y' = y^2$ con $y(0) = y_0 > 0$, la solución es $\phi(t) = \frac{y_0}{1-y_0t}$, que solo existe en el intervalo $(-\infty, 1/y_0)$ y "explota" en $t = 1/y_0$. Este es un ejemplo de una **solución maximal**.

Es importante notar que las soluciones pueden no ser únicas si $f(y_0)=0$ en la condición inicial.

\[
  \text{POR COMPLETAR}
\]

\section{Problema de Cauchy: existencia y unicidad}

		Sea $\Omega \subset \R^{n+1}$ un conjunto abierto, que por lo general será de la forma $ \Omega = I \times \widetilde{\Omega}$, con $I \subset \R$ un intervalo abierto y $\widetilde{\Omega}\subset \R^{n}$ un conjunto abierto. Dada una función
		\[ f \in C (\Omega;\R^n) \]
		\noindent consideramos nuevamente el problema de Cauchy
		\begin{align*} (PC) \ \begin{cases}
			 y'(x) = f(x,y(x)), \quad \forall x \in I \text{ con } & x_0 \in I \text{ y} \\
			 y(x_0) = y_0, & y \in \R^n
		\end{cases} \end{align*}

		\noindent El (PC) puede ser formulado de manera equivalente, pero "relativamente" más débil:

		\begin{lemma}
			Sea $\Omega \subset \R^{n+1}$ un conjunto abierto de la forma $\Omega = I \times \widetilde{\Omega},\ I \subset \R$ intervalo abierto y $\widetilde{\Omega} \subset \R^n$ conjunto abierto. Dados $x_0 \in I,\ y_0 \in \R^n$ y $f \in C(\Omega;\R^n)$, una función $\varphi : I \to \R^n$ es solución de (PC) si y sólo si:
			\begin{enumerate}
				\item $\varphi \in C(I;\R^n)$;

				\item $(x,\varphi(x)) \in \Omega \quad \forall x \in I$;

				\item $\varphi (x) = y_0 + \int_{x_0}^{x} f(s,\varphi(s)) ds \quad \forall x \in I$.
			\end{enumerate}
			(i, ii y iii es formulación integral del (PC)).
		\end{lemma}

		\begin{remark}
			La formulación integral nos permite estudiar el (PC) desde una perspectiva más abstracta. Supongamos por ahora que
			\[ \Omega = \R^{n+1},\ I = \R,\ \widetilde{\Omega} = \R^n \text{ y } f \in C(\R^{n+1};\R^n). \]
		\end{remark}

		\noindent Dados $y_0 \in \R^n$ y $x_0 \in \R$, consideramos la aplicación $T: C(\R;\R^n) \to C(\R;\R^n)$.
		\[ T(\varphi)(x) = y_0 + \int_{x_0}^{x} f(s,\varphi(s)) ds \quad \forall x \in \R \]
		
		\noindent Por el lema precedente, es evidente que $\varphi \in C(\R;\R^n)$ es solución de (PC) si y sólo si $T(\varphi) \equiv \varphi$ (i.e., $\varphi$ es punto fijo de $T$).

Sea $\Omega \subset \R^{n+1},\ \Omega = I \times \widetilde{\Omega}$ con $I \subset \R$ intervalo abierto y $\widetilde{\Omega} \subset \R^n$ conjunto abierto.
\[ (PC) \ \begin{cases}
	y'(x) = f(x, y(x))\quad \forall x \in I, \\
	y(x_0) = y_0 \quad (x_0 \in I, y_0 \in \R^n)
\end{cases} \]

\begin{enumerate}
	\item Resolver localmente el problema (PC) corresponde a encontrar un intervalo $J \subset I$ y una función $\varphi \in C^1 (J;\R^n)$ tales que:
	\[ x_0 \in J,\ (x,\varphi(x)) \in \Omega \ \forall x \in J,\ \varphi'(x) = f(x,\varphi(x))\ \forall x \in J; \]

	\item Si $J_1,\ J_2 \subset I$ son intervalos abiertos que contienen a $x_0$, y $\phi_1 \in C^1(J_1;\R^n)$ y $\phi_2 \in C^1(J_2; \R^n)$ son soluciones locales de (PC), decimos que $\phi_2$ extiende a $\phi_{1}$ es una restricción de $\phi_{2}$;

	\item Una solución es \underline{maximal} cuando no admite extensiones;

	\item Una solución local, definida sobre $J \subset I$, es global si $J = I$.
\end{enumerate}

\noindent \textbf{Recuerdo.} Dados $a<b$ números reales, el espaacio vectorial
\[ C([a,b];\R^n) \]
\noindent dotado de la norma $\| \varphi \|_{\infty} = \max_{x \in [a,b]} |\varphi(x)| \quad \forall \varphi \in C([a,b];\R^n)$, es un espacio de Banach.

\begin{itemize}
	\item Todo sub-conjunto cerradeo de $(C([a,b];\R^n);\| \cdot \|_{\infty})$ es un espacio métrico completo.

	\item Todo sub-conjunto vectorial cerrado de $(C([a,b];\R^n);\| \cdot \|_{\infty})$ es un espacio de Banach.
\end{itemize}


\subsection{Aplicaciones contractivas: teorema del punto fijo de Baanch}

\begin{definition}[aplicación contractiva]
	Sea $(X,d)$ un espacio métrico completo. Decimos que una aplicación $T: X \to X$ es \underline{contractiva} si existe $\alpha \in [0,1)$ tal que
	\[ d(T(x),T(y)) \leq \alpha d(x,y) \quad \forall x,y \in X \]
\end{definition}

\begin{theorem}[punto fijo de Banach]
	Sea $(X,d)$ un espacio métrico completo y $T: X \to X$ una aplicación contractiva. Entonces, existe un único $\hat{x} \in X$ tal que $T(\hat{x}) = \hat{x}$.
\end{theorem}


Funciones Lipschitz: $\Omega \subset \R^{n+1},\ (x,y) \in \Omega,\ x \in \R$, y $y \in \R^n$

\begin{definition}[función globalmente Lipschitz]
	Sea $\Omega \subset \R^{n+1}$ abierto, y $f : \Omega \to \R^n$ una función. Decimos que $f$ es globalmente Lipschitz respecto a la variable $y$ en $\Omega$ si existe una constante $L>0$ tal que
	\[ |f(x,y_1) - f(x,y_2)| \leq L | y_1 - y_2 | \quad \forall (x,y_1), (x,y_2) \in \Omega \]
\end{definition}

\begin{note}
	Lip$(y;\Omega)$ denota el espacio vectorial de todas las funciones $f:\Omega\to\R^n$ que son globalmente Lipschitz respecto a $y$ en $\Omega$.
\end{note}

\begin{definition}[función localmente Lipschitz]
	Sea $\Omega \subset \R^n$ abierto. Se dice que una función $f: \Omega \to \R^n$ es \underline{localmente} Lipschitz respecto a la variable $y$ en $\Omega$ si: para cualquier punto $(\overline{x}, \overline{y})\in\Omega$, existe $\varepsilon>0$ y una constante $L>0$ tales que $B((\overline{x},\overline{y}),\varepsilon) \subset \Omega$, y 
	\[ |f(x,y_1) - f(x,y_2)| \leq L |y_1 - y_2| \quad \forall (x,y_1),(x,y_2) \in B((\overline{x},\overline{y},\varepsilon). \]
\end{definition}

\begin{note}
	$\text{Lip}_{\textit{loc}}(y;\Omega)$.
\end{note}

\begin{prop}~
	\begin{enumerate}
		\item Si $f \in \text{ Lip}(y;\Omega)$, entonces $f$ es uniformemente continua respecto a $y$ en $\Omega$: $(\forall \varepsilon > 0)(\exists \delta > 0)(\forall (x,y_1),(x,y_2) \in \Omega,$
		\[ |y_1-y_2|\leq\delta \implies |f(x,y_1)-f(x,y_2)| \leq \varepsilon \]

		\item Si $f\in \text{ Lip}_{\textit{loc}}(y;\Omega)$, entonces $f$ es continua respecto a la variable $y$ en $\Omega$: para todo $(\overline{x},\overline{y})\in\Omega$:
		\[ (\forall \varepsilon > 0)(\exists \delta > 0)\ |y-\overline{y}|\leq \delta \implies (\overline{x},\overline{y})\in\Omega \text{ y } |f(\overline{x},\overline{y}) - f(\overline{x},y)| \leq \varepsilon \]
	\end{enumerate}
\end{prop}

\begin{remark}
	En general, Lip$(y;\Omega) \not\subseteq C(\Omega;\R^n)$. Por ejemplo,
	\[ f(x,y) = \begin{cases}
		0 \text{ si } x \leq 0 \\
		y \text{ si } x > 0
	\end{cases} \]
	pertenece a Lip$(y;\R^2)$, pero es discontinua en el conjunto $\{ (x,y) \in \R^2 \ | \ x = 0 \}$. Recíprocamente, la continuidad ("en pareja") de una función no implica ningún tipo de Lipschitzianidad (local o global).
\end{remark}

\begin{eg}
	$\hat{f}(x,y) = \sqrt{|y|}\quad \forall (x,y) \in \R^2$. Vemos que no es localmente Lipszhitz: $\hat{f} \not\in \text{ Lip}_{\textit{loc}}(y;\R^2)$: por contradicción, debiese existir $\varepsilon > 0$ tal que $f \in \text{ Lip}(y; B((0,0),\varepsilon))$. Por ende, existe una constante $L>0$ tal que
	\begin{align*}
		\left|f(0,0)-f \left(0,\frac{\varepsilon}{n}\right)\right| & \leq L \cdot \frac{\varepsilon}{n} \quad \forall n \geq 2, \\
		& \iff \sqrt{\frac{\varepsilon}{n}} \leq \frac{L \varepsilon}{n} \\
		& \iff \sqrt{n} \leq L \sqrt{\varepsilon} \quad \forall n \geq 2,
	\end{align*}
	\noindent lo cual es absurdo!
\end{eg}

\begin{theorem}
	Sea $\Omega\subset\R^{n+1}$ un conjunto abierto y
	\[ f = (f_1,f_2,\dots,f_n):\ \Omega \to \R^n \]
	\noindent una función tal que las derivadas parciales $\frac{\partial f_i}{\partial y_j}\ \forall i,j \in \{1,\dots,n\}$ existen y son continuas en $\Omega$. Entonces:
	\begin{enumerate}
		\item $f \in \text{Lip}_{\textit{loc}}(y;\Omega)$;

		\item Si, además, $\Omega$ es convexo, $f \in \text{Lip}(y;\Omega)$ si y sólo si
		\[ \sup_{(x,y)\in\Omega} \left| \frac{\partial f_i}{\partial y_j}(x,y) \right| < \infty \quad \forall i,j \in \{ 1,\dots, n \} \]
	\end{enumerate}
\end{theorem}

\begin{remark}
	Consideramos las funciones $f_1,f_2,f_3:\R^2 \to \R$ definidos por:
	\[ f_1(x,y)  = \frac{1}{1+y^2},\ f_2(x,y) = \frac{x}{1+y^2},\ f_3(x,y)= \begin{cases}
		xy \quad \text{si } x>0; \\
		y \quad \text{ si } x\leq 0.
	\end{cases} \]
	Se puede demostrar que:
	\begin{enumerate}
		\item $f_1 \in \text{Lip}(y;\R^2)$;
		
		\item $f_2 \in \text{Lip}(y;\R^2)$, pero $f_2 \in \text{Lip}(y;\Omega)$ para cualquir dominio $\Omega \subset \R^2$ que sea acorada en al dirección de $x$;

		\item $f_3 \in \text{Lip}(y;\R^2)$ y $f_3 \in \text{Lip}(y;\Omega)$ para cualquier dominio $\Omega \subset \R^2$ que sea acotado en la dirección de $x$. No obstante, $\frac{\partial f_3}{\partial y}$ \underline{no} es continua en $\R^2$.
	\end{enumerate}
\end{remark}

\begin{theorem}
	Sea $\Omega \subset \R^{n+1}$ un conjunto abierto. Si $f \in \text{Lip}_{\text{loc}}(y;\Omega)$ y $K \subset \Omega$ es un conjunto compacto tal que $\sup_{(x,y) \in K} |f(x,y)| < \infty$, entonces $f \in \text{Lip}(y;K)$.
\end{theorem}

\begin{theorem}[Picard-Lindelöf]
	Sea $\Omega \subset \R^{n+1}$ un conjunto abierto y $f\in C(\Omega; \R^n) \cap \text{Lip}_{\text{loc}}(y;\Omega)$. Entonces, para cada $(x_0,y_0) \in \Omega$, existe un $\delta > 0$ tal que, si denotamos
	\[ I_{\delta} = [x_0 - \delta, x_0 + \delta], \]
	\noindent el problema de Cauchy
	\[ (PC)\ \begin{cases}
		y'(x) = f(x,y(x))\quad \forall x \in I_{\delta}, \\
		y(x_0) = y_0
	\end{cases} \]
	\noindent admite una única solución.
\end{theorem}

\begin{proof}
	Como $\Omega$ es abierto y $(x_0,y_0) \in \Omega$, existen $a_0>0$ y $b_{0}>0$ tal que, si ponemos
	\[ R = [x_0-a_0,x_{0}-a_{0}] \times \overline{B(y_0,b_0)}, \]
	entonces $R \subset \Omega$ y $f \in \text{Lip}(y;R)$.
	\begin{itemize}
		\item Sea $L>0$ una constante de Lipschitsz para $f$ en $R$ (respecto a $y$):
		\[ |f(x,y_{1}) - f(x,y_{2})| \leq L |y_{1} - y_{2}|\quad \forall (x,y_{1}),(x,y_{2})\in R. \]

		\item Denotemos por $M = \max_{(x,y)\in R} |f(x,y)|$.
	\end{itemize}
	Fijemos un número $\delta>0$ tal que 
	\[ 0 < \delta < \min \left\{a_0, \frac{b_0}{M}, \frac{1}{L} \right\} \]
	y consideremos el conjunto:
	\[ X = \{ \varphi \in C(I_{\delta};\R^n) \ \big{|} \ |\varphi(x) - y_{0}| \leq b_{0} \quad \forall x \in I_{\delta} \} \]  
	donde $(X,d)$ es un espacio métrico completo con la distancia usual en $C(I_{\delta},\R^n)$. \\
	Definimos la aplicación $T: X \to C(I_{\delta};\R^n)$ por
	\[ T(\varphi)(x) = y_{0} + \int_{x_{0}}^{x} f(s,\varphi(s))ds \quad \forall \varphi \in X,\ \forall x \in I_{\delta}. \]
	Notemos que, dado $\varphi \in X, \ \varphi$ es solución de (PC) en $I_{\delta}$ si y sólo si $\varphi$ es un punto fijo de la aplicación $T$ en $X$.
	\[ \varphi \in C(I_{\delta};\R^n)\ : \ (x,\varphi(x)) \in \Omega \quad \forall x \in I_{\delta};\ \varphi(x_{0})=y_{0} \]
	\textbf{PDQ:} $|\varphi(x)-y_{0}| \leq b_{0}\quad \forall x \in I_{\delta}$.

	\begin{proof}
		Por contradicción, supongamos que $\exists \hat{x} \in I_{\delta}:\ |\varphi(x)-y_{0}| > b_{0}$. Por continuidad, existe $\hat{x}_0 \in I_{\delta}$ tal que 
		\[ |\varphi(\hat{x}_{0} - y_{0}| = b_{0} \text{ y } |\varphi(x) -y_{0}\quad x \in (x_{0},\hat{x}_0). \]
		\begin{align*}
			b_{0} & = |\varphi(\hat{x}_0)-y_{0}| \\ 
			&lc= \left| \int_{x_0}^{\hat{x}_0} f(s,\varphi(s))ds \right| \\
			& \leq \left| \int_{x_0}^{\hat{x}_0 } |f(s,\varphi(s))| ds \right| \\
			& \leq M |\hat{x}_0 - x_{0}| \leq M \delta < b_0 
		.\end{align*}
	\end{proof}

	\noindent Dada $\varphi \in X$, ciertamente $T(\varphi) \in C(I_{\delta};\R^n)$. Además, $\forall x \in I_{\delta}$,
	\begin{align*}
		|T(\varphi)(x)-y_{0}| & = \left| \int_{x_{0}}^{x} f(s,\varphi(s))ds \right| \\ 
		& \leq \left| \int_{x_{0}}^{x} |f(s,\varphi(s))ds| \right| \\
		& \leq M |x-x_{0}| \leq M\delta < b_0
	.\end{align*}
	entonces, $T(\varphi) \in X$.
	Demostremos ahora que $T$ es una contracción: dados $\varphi,\psi \in X$, 
	\begin{align*}
		|T(\varphi)(x)-T(\psi)(x)| & = \left| \int_{x_{0}}^{x} (f(s,\varphi(s))-f(s,\psi(s)))ds \right| \\
		& \leq \left| \int_{x_{0}}^{x} |f(s,\varphi(s))-f(s,\psi(s))| ds \right| \\
		& \leq L \left| \int_{x_{0}}^{x}|\varphi(s)-\psi(s)|ds \right| \\
		& \leq L|x-x_{0}| \| \varphi - \psi_0 \|_{\infty} \leq L \delta \| \varphi - \psi \|_{\infty},
	\end{align*}
	con $L\delta<1$. Luego, $T: X \to X$ es una contracción, y entonces admite un único punto fijo $\hat{\varphi} \in X$, que constituye la única solución del (PC) en $I_{\delta}$. 
\end{proof}
 
i\begin{theorem}[Picard-Lindelöf para EDOs de orden superior]
  Sea $\Omega \subset \R^{n+1}$ un conjunto abierto y 
  \[g \in C(\Omega;\R)\cap \text{lip}_{\textit{loc}}(y,y',\ldots, y^{n-1};\R).\] 
  dado cualquier punto $(x_0,y_0,y'_0,\ldots,y_0^{n-1})\in \Omega$, existe $\delta > 0$ tal que, si ponemos $I_{\delta}=[x_0 -\delta, x_0+ \delta]$, el problema de Cauchy
  \[
    \begin{cases}
      y^{m}(x)=g(x,y(x),y'(x),\ldots,y^{n-1}(x)) \quad \forall x \in I_{\delta} \\
      y(x_0)=y_0, y'(x_o)=y'_o,\ldots, y^{n-1}(x_o)=y_0^{n-1},
    \end{cases}
  \]
       
  admite una única solución.

\end{theorem}

\begin{example}[Ecuaciones integrales] 
    Sea $a<b$ números reales, $I=[a,b], f\in C(I;\R), k\in C(I\times I:\R)$ y $\lambda \in \R$ elementos dados. El problema de Volterra (de segunda especie) consiste en hallar una función $\varphi_{\lambda}\in C(I;\R)$ tal que 
  \[
    \varphi_{\lambda}(x)=f(x)+\lambda\int_{a}^{x}k(x,t)\varphi_{\lambda}(t)dt.\quad \forall x \in [a,b].
  \]
\end{example}
\subsection{Un poco más de análisis Funcional.}
Recordamos que en cualquier espacio métrico, un sub-conjunto es compacto si y sólo si es secuecialmente compacto.

\begin{definition}
  Sean $a<b$ números reales, $I=[a,b]$ Dada una familia de funciones $  F\subset C(I;\R)$, decimos que
  \begin{enumerate}
    \item $F$ es relativamente compacto si $\overline{F}$ es compacto, vale decir para cualquier sucesión $(\varphi_n)_{n\in \N} \subset F$, existen $\hat{\varphi} \in C(I;\R)$ y una sub-sucesión $(\varphi_{n_k})_{k\in \N}$ tales que $\varphi_{n_k} \xrightarrow{k\to \infty}\hat{\varphi}$ es $ C(I;\R)$
    \item $F$ es equicontinua si $(\forall \epsilon >0)(\exists \delta > 0)(\forall x_1,x_2 \in I) \quad |x_1-x_2|<\delta \Rightarrow |\varphi(x_1)-\varphi(x_2)|<\epsilon \forall \varphi \in F$ 
    \item $F$ es acotada si $\sup_{\varphi \in F}|\varphi|_{\infty} < \infty $ 
  \end{enumerate}
\end{definition}

\begin{theorem}[Arzelà-Ascoli] 
  Sean $a<b$ números reales y $I=[a,b]$. Dada una familia $F\subset C(I;\R)$, se tiene que $F$ es relativamente compacto si y sólo si $F$ es equicontinua y acotada.
\end{theorem}

\begin{theorem}[Punto fijo de Brower] 
  Sea $K\subset \R^n$ un conjunto compacto, conexo y no-vácio. Sea $T:K\rightarrow K$ una aplicación continua. Entonces existe, al menos, un punto fijo de $T$ en $K$
\end{theorem}

\begin{theorem}[Punto fijo de Schauder]
  Sea $\mathcal{X}$ un espacio de Banach sobre $\R$, y $K\subset \mathcal{X}$ un conjunto conexo, cerrado, acotado y no vacío. Consideremos una aplicación $T:K\rightarrow K$  continua y tal que $T(K)$ es relativamente compacto, Entonces existe, al menos, un punto fijo de $T$ en $K$.
\end{theorem}

\begin{remark}
  $T(k)$ es relativamente compactio si y sólo si $\overline{T(K)}$ es compacto, si y sólo si cualquier sucesión $(X_n)_{n\in \N}\subset \mathcal{X}$ resulta que $(T(X_n))_{n\in \N}$ admite una sub-sucesión convergente (a un límite que no necesariamente pertenece a $T(K)$).
\end{remark}

\begin{theorem}[Peano] Sea $\Omega \subset \R^{n+1}$ un conjunto abierto y $f \in C(\Omega;\R^n)$. Entonces, para cualquier $(x_0,y_0)\in \Omega$, existe $\delta>0$ tal que, si ponemos $I_{\delta}=[x_o-\delta,x_o+\delta]$, el problema de Cauchy
  \[
    \begin{cases}
      y´(x)=f(x,y(x)) \quad x \in I_{\delta} \\ y(x_0)=y_0
    \end{cases}
  \]
  Admite (al menos) una solución
\end{theorem}

\begin{proof}
  Como $\Omega$ es abierto, existen $a_0>0$ y $b_0>0$ tales que, si denotamos $R=[x_0-a_0,x_0+a_0]\times \overline{B(y_0,b_0)}$, entonces $R \subset \Omega$. Sea $M=\max_{(x,y)\in \R}|f(x,y)|$ y tomamos $\delta > 0$ tales que $0 < \delta < \min\left\{a_0, \frac{b_0}{M}\right\}$. Definamos 
  \[  K=\{\varphi \in C(I_{\delta};\R^n)| \varphi(x_0)=y_0;\quad |\varphi(x)-y_0|\leq b_o \quad \forall x \in I_{\delta}\}\]
  que es no vacío, conexo, conexo y acotado. Definamos ahroa la aplicación $T:K\rightarrow C(I_{\delta};\R^n)$ mediante la fórmula 
  \[ T(\varphi)(x)=y_0 + \int_{x_0}^{x} f(s,\varphi(s))ds \quad \forall \varphi \in K, \quad \forall x \in I_{\delta}\]
  Notamos que $T(\varphi)\in C(I_{\delta};\R^n) \quad \forall \varphi \in K$. Además, si $\varphi \in K$, se tiene que $(x,\varphi(x)\in \R, \quaf \forall x \in I_{\delta}$ y así:
  \[ |T(\varphi)(x)-y_0|\leq \left| \int_{x_0}^{x}|f(s,\varphi(s))|ds \right| \leq M|x-x_0|\leq M\delta -b_0 \quad \forall x \in I_{\delta} \]
  y como $T(\varphi)(x_0)=y_0$, entonces $T(\varphi)\in K\quad \forall \varphi \in K.$. Ahora, afirmamos que $T:K\rightarrow K$ es continua, sea $\epsilon >0$. Como $f$ es uniformemente continua en $\R$, exite $\delta_0 >0$ tal que $:\forall(s,y_1),)(s,y_2)\in \R$, 
  \[|y_1-y_2|\leq \delta_0 \Rightarrow |f(s,y_1)-f(s,y_2)|\leq \frac{\epsilon}{\delta}\]
  Por ende, si $\varphi_1,\varphi_2 \in K$ son tales que $|\varphi_1,\varphi_2|_{\infty}\leq \delta_0$, entonces $|\varphi_1(s)-\varphi_2(s)|\leq \delta_0\quad \forall s \in I_{\delta}$. Entonces 
  \[ |f(s,\varphi_1(s))-f(s,\varphi_2(s))|\leq \frac{\epsilon}{\delta}\quad \forall s \in I_{\delta} \]
  Luego,
  \[|t(\varphi_1)(x)-T(\varphi_2)(x)|\leq \left|\int_{x_0}^{x}|f(s,\varphi_1(s))-f(s,\varphi_2(s))|\right| \leq \frac{\epsilon}{\delta}|x-x_0|\leq \epsilon \quad \forall x \in I_{\delta}\]
  Entonces $||T(\varphi_1)-T(\varphi_2)||_{\infty}\leq \epsilon \quad \forall \varphi_1,\varphi_2\in K$ tales que $||\varphi_1 -\varphi_2||_{\infty} \leq \delta_0$ con esto, demostramos que la familia $T(K)\subset C(I_{\delta};\R^n)$ es equicontinua. Dados $x_1,x_2 \in I_{\delta}$ y $\varphi \in K$, notamos que \[|T(\varphi)(x_1) - T(\varphi)(x_2)| \leq \left|\int_{x_1}^{x_2}|f(s,\varphi(s))|ds \right|\leq M|x_1 - x_2|  \] Así, dado $\epsilon > 0$, escogemos $\delta_1 = \frac{\epsilon}{M}$, de modo que \[ (\forall x_1,x_2 \in I_{\delta}) \quad |x_1 - x_2| \leq \delta_1 \Rightarrow |T(\varphi)(x_1) - T(\varphi)(x_2)| \leq \epsilon \quad \forall \varphi \in K \] Entonces, demostramos que la fórmula $T(K)$ es acotada: \[ \sup_{\varphi \in K} ||T(K)||_{\infty} < \infty \] Dado $\varphi \in K$ y $x \in I_{\delta}$, se tiene 
    \[
      \begin{align*}
        |T(\varphi)(x)| &\leq \left| y_0 + \int_{x_0}^{x} f(s,\varphi(s)) ds \right| \leq |y_0| + \left|  \int_{x_0}^{x} f(s,\varphi(s)) ds \right| \\
        & \leq |y_0| + M|x - x_0| \leq  |y_0| + M\delta \leq |y_0| + b_0
      \end{align*}
    \] 
  Por lo tanto \[ ||T(\varphi)||_{\indty} \leq |y_0| + b_0 \quad \forall \varphi \in K\] El teorema de Arzelà-Ascoli asegura entonces que $T(K)$ es relativamente compacto. Luego, se cumplen todas las hipótesis del teorema de punto fijo de Schauder, y así,  $(\exists \hat{\varphi} \in K) \quad T(\hat{\varphi}) = \hat{\varphi}$, entonces: \[ \hat{\varphi} \in C(I_{\delta};\R^n) \text{ es solución de } (PC) \]
\end{proof}
\begin{theorem}
  Sea $\Omega \subseteq \mathbb{R}^{n+1}$ un conjunto abierto y $g \in C(\Omega, \mathbb{R})$.Para cualquier $(x_0, y_0, y_0', \dots, y_0^{(n-1)}) \in \Omega$, existe $\delta > 0$ tal que, si ponemos $I_\delta = [x_0 - \delta, x_0 + \delta]$, el problema de Cauchy
  \[
    (PC) \quad \begin{cases}
    y^{(n)}(x) = g(x, y(x), y'(x), \dots, y^{(n-1)}(x)) & \forall x \in I_\delta, \\
    y(x_0) = y_0, y'(x_0) = y_0', \dots, y^{(n-1)}(x_0) = y_0^{(n-1)}
    \end{cases}
  \]
  admite, al menos, una solución.
\end{theorem}
\[
  \text{PIZARRA POR COMPLETAR}
\]
\begin{example}[Iteraciones de punto fijo.]
  Consideramos el siguiente problema de Cauchy (PC):
\[
  (PC) \quad \begin{cases}
  y'(t) = y(t) + t - 1 & \forall t \in \R, \\
  y(0) = 1,
  \end{cases}
\]
cuya (única) solución es $y(t) = e^t - t \quad \forall t \in \R$. La formulación integral de (PC) es:
\[
  Y(t) = 1 + \int_{0}^{t} [Y(s) + s - 1] \,ds \quad \forall t \in \R.
\]
Consideremos la sucesión de funciones $(\varphi_n)_{n \in \N} \subset C(\R; \R)$ dada por:
\[
  \varphi_0(t) = 1 \quad \forall t \in \R, \quad
  \varphi_{n+1}(t) = 1 + \int_{0}^{t} [\varphi_n(s) + s - 1] \,ds \quad \forall n \in \N, \forall t \in \R. 
\]

Calculamos los primeros términos:
\[
  \begin{align*}
    \varphi_1(t) &= 1 + \int_{0}^{t} (\varphi_0(s) + s - 1) \,ds = 1 + \int_{0}^{t} (1 + s - 1) \,ds = 1 + \int_{0}^{t} s \,ds \\
    &= 1 + \frac{t^2}{2} = 1 + \frac{t^2}{2!} \quad \forall t \in \mathbb{R}. \\
    \varphi_2(t) &= 1 + \int_{0}^{t} (\varphi_1(s) + s - 1) \,ds = 1 + \int_{0}^{t} \left(1 + \frac{s^2}{2!} + s - 1\right) \,ds \\
    &= 1 + \int_{0}^{t} \left(\frac{s^2}{2} + s\right) \,ds = 1 + \frac{t^3}{6} + \frac{t^2}{2} = 1 + \frac{t^2}{2!} + \frac{t^3}{3!} \quad \forall t \in \R.
  \end{align*}
\]
Por inducción, se puede demostrar que:
\[ 
  \varphi_n(t) = 1 + \frac{t^2}{2!} + \frac{t^3}{3!} + \cdots + \frac{t^{n+1}}{(n+1)!} = 1 + \sum_{k=2}^{n+1} \frac{t^k}{k!}.
\]
Tomando el límite cuando $n \to \infty$:
\[ 
  \lim_{n \to \infty} \varphi_n(t) = 1 + \sum_{k=2}^{\infty} \frac{t^k}{k!} = \left(\sum_{k=0}^{\infty} \frac{t^k}{k!}\right) - t = e^t - t \quad \forall t \in \R. 
\]
Con convergencia uniforme en cada intervalo compacto de $\R$.
\end{example}

\begin{example} Consideremos el siguiente problema de Cauchy (PC):

	\[ (PC) \ \begin{cases}
		y'(t) = \frac{t+2}{t^2+(y(t))^2} \quad \forall t \in I \subset \R, \\
		y(0) = 1
	\end{cases} \]
	Sea $\Omega = \R \setminus \{(0,0)\}$ y $f(x,y) = \frac{x+2}{x^2 + y^2} \quad \forall (x,y) \in \Omega$. Como $f \in C^{\infty}(\Omega) \implies f \in \text{Lip}_{loc}(y;\Omega)$, y así, existe $\delta > 0$ tal que (PC) admite una única solución en $I_{\delta} = [-\delta,\delta]$, que denotamos por $\varphi \in C^{1}(I_{\delta};\R)$. De hecho, $\varphi \in C^{\infty}((\delta,\delta);\R)$ y podemos escribir un polinomio de MacLaurin de orden $2$ con resto de Peano:
	\[ \varphi(t) = \varphi(0) + \varphi'(0)t + \frac{\varphi''(0)}{2}t^2 + o(t^2) \]
	para $t \to 0$. Sabemos que $\varphi(0) = 1$ y $\varphi'(0) = \frac{0+2}{0+1}=2$. Además, derivando $(PC)_{(1)}$ se obtiene:
	\[ \varphi''(t) = \frac{t^2 + (\varphi(t))^2 - (t+2)(2t+2\varphi(t)\varphi'(t))}{(t^2 + (\varphi(t))^2)^2} \quad \forall t \in \mathring{I}_{\delta}, \]
	con lo que: $\varphi''(0) = -7$. Luego:
	\[ \varphi(t) = 1 + 2t - \frac{7}{2}t^2 + o(t^2) \]
	para $t \to 0$.
\end{example}

\subsection*{ Unicidad global y solución global del problema de Cauchy}

\begin{lemma}[Grönwall]
	Sean $x_0 < x_1$ números reales, $h \in \R$ y $u,k \in C([x_0,x_1];\R)$ funciones tales que $k(x) \geq 0 \quad \forall x \in [x_0,x_1]$. Entonces,
	\begin{enumerate}
		\item[(A)] Si $u(x) \leq h + \int_{x_{0}}^{x} k(s)u(s) ds \quad \forall x \in [x_0,x_1]$, entonces
		\[ u(x) \leq h\cdot \exp\left( \int_{x_{0}}^{x} h(s)ds \right) \quad \forall x \in [x_{0},x_{1}] \]

		\item [(B)] Si $u(x) \leq h + \int_{x}^{x_{1}}h(s)u(s)ds \quad \forall x \in [x_{0},x_{1}]$, entonces:
		\[ u(x) \leq h \cdot \exp \left( \int_{x}^{x_{1}}h(s)ds \right) \quad \forall x \in [x_{0},x_{1}] \]
	\end{enumerate}
\end{lemma}
\begin{proof}(sólo apartado (A))
	Definamos la función 
	\[ v(x) = \int_{x_{0}}^{x} k(s)u(s)ds \quad \forall x \in [x_{0},x_{1}], \]
	y así, $v \in C^1([x_{0},x_{1}];\R),\ v(x_{0}) = 0$ y $v'(x) = k(x)u(x) \quad \forall x \in (x_{0},x_{1})$. Como $u(x) \leq h + v(x) \quad \forall x \in [x_{0},x_{1}]$, entonces
	\begin{align*}
		& h(x)u(x) \leq h\cdot k(x) + v(x)k(x) \quad \forall x \in [x_{0},x_{1}] \\
		\iff \ & v'(x) \leq h\cdot k(x) + k(x)v(x) \quad \forall x \in [x_{0},x_{1}]
	.\end{align*}
	Luego:
	\begin{align*}
		& \exp \left( - \int_{x_{0}}^{x} h(s)ds \right) v'(x) - \exp \left(- \int_{x_{0}}^{x}h(s)ds \right)k(x)v(x) \\
		& \leq h\cdot k(x) \exp \left( - \int_{x_{0}}^{x} k(s)ds \right) \\
		\iff \ & \frac{d}{dx} \left[ \exp \left(-\int_{x_{0}}^{x}k(s)ds\right)v(x) \right] \leq -h \frac{d}{dx} \left[ \exp \left( - \int_{x_{0}}^{x} h(s)ds \right) \right]
	.\end{align*}
	Integrando en $[x_{0},x]$, con $x \in (x_{0},x_{1}]$, se obtiene:
	\begin{align*}
		& \exp \left( -\int_{x_{0}}^{x}k(s)ds \right) v(x) \leq h \left[ 1 - \exp \left( -\int_{x_{0}}^{x} k(s)ds \right) \right] \\
		\implies \ & v(x) \leq h \cdot \exp \left( \int_{x_{0}}^{x} k(s)ds \right) -h \\
		\iff \ & u(x) \leq h + v(x) \leq h \cdot \exp \left( \int_{x_{0}}^{x}k(s)ds \right) \quad \forall x \in [x_{0},x_{1}]
	\end{align*}
\end{proof}

\begin{remark}~
	\begin{enumerate}
		\item Si $h=0$, el lema de Grönwall asefura que $u(x) \leq 0 \quad \forall x \in [x_{0},x_{1}]$;

		\item Es posible utilizar una única formulación que agrupe las dos condiciones del lema:
		\begin{align*}
			& u(x) \leq h + \left| \int_{x_{0}}^{x}k(s)u(s)ds \right| \quad \forall x,x_{0}\in I \\
			\implies \ & u(x) \leq h \cdot \exp \left( \left| \int_{x_{0}}^{x} k(s)ds \right| \right)
		.\end{align*}
	\end{enumerate}
\end{remark}

\begin{theorem}[unicidad global]
	Sea $\Omega \subset \R^{n+1}$ un conjunto abierto, $(x_{0},y_{0}) \in \Omega$ y $f \in C(\Omega;\R^n) \cap \text{Lip}_{loc}(y;\Omega)$. Si $(I_1,\varphi_1)$ y $(I_2,\varphi_{2})$ son soluciones locales del problema de Cauchy
	\[ (PC) \ \begin{cases}
		y'(x) = f(x,y(x)) \quad \forall x \in I \subset \R, \\
		y(x_{0}) = y_{0}
	\end{cases} \]
	entonces $\varphi_{1}(x) = \varphi_{2}(x)\quad \forall x \in I_1 \cap I_2$ (acá, $I_1,I_{2} \subset \R$ son intervalos que conienen a $x_{0}$).
\end{theorem}
\begin{proof}
	Sabemos que $(I_1 \cap I_{2}) \setminus \{x_{0}\} \neq \varnothing$, y así, tomemos $x_{1} \in (I_{1} \cap I_{2}) \setminus \{x_{0}\}$, con $x_{1} > x_{0}$. Definamos el conjunto:
	\[ K = \{ (s,\varphi_{1}(s)) \ \big| \ s \in [x_{0},x_{1}] \} \cup \{ (s,\varphi_{2}(s) \ \big| \ s \in [x_{0},x_{1}] \} \] 
	Luego, $K \subset \Omega$ es compacto. Como $f$ es continua en $\Omega$, resulta que: $\sup_{(x,y) \in K} |f(x,y)| < \infty \implies f \in \text{Lip}(y;K)$, y sea $L_k > 0$ una constante de Lipschitz (global) para $f$, respecto a su segunda variable en $K$. Sabemos que $\varphi_{i}(x) = y_0 + \int_{x_{0}}^{x} f(s,\varphi(s))ds \quad \forall x \in [x_0,x_{1}],\ \forall i \in \{1,2\}$. Entonces,
	\begin{align*}
		|\varphi_{1}(x) - \varphi_{2}(x)| &= \left| \int_{x_{0}}^{x} [f(s,\varphi_{1}(s)) - f(s,\varphi_{2}(s))]ds \right| \\
		& \leq \int_{x_{0}}^{x} \big|f(s,\varphi_{1}(s)) - f(s,\varphi(s))\big| ds \\
		& \leq L_K \int_{x_{0}}^{x} \big| \varphi_{1}(s) - \varphi_{2}(s) \big| ds \quad \forall x \in [x_{0},x_{1}]
	.\end{align*}
	Así (gracias al lema de Grönwall), $\big| \varphi_{1}(x) - \varphi_{2}(x) \big| = 0 \quad \forall x \in [x_{0},x_{1}] \implies \varphi_{1}(x) = \varphi_{2}(x) \quad \forall x \in [x_{0},x_{1}]$.
\end{proof}

\subsection{Prolongación de soluciones para EDOs}

Dada $(x_0, y_0) \in \Omega$, sabemos que existen $a_0, b_0 > 0$ tales que
\[ R \coloneqq [x_0 - a_0, x_0 + a_0] \times \overline{B(y_0, b_0)} \subseteq \Omega \quad (\text{inclusión estricta}) \]
Sea $M \coloneqq \max_{(x,y) \in R} |f(x,y)|$ y $L_{R} > 0$ una constante de Lipschitz (local) para $f^{(x,y)\in R}$ en $R$ respecto a $y$.

En el Teorema de Picard-Lindelöf se escogía $\delta > 0$ tal que
\[  0 < \delta \leq \min\{a_0, b_0/M, 1/L_{R}\} \]
De hecho, se puede llegar a la misma conclusión seleccionando $\delta > 0$ tal que
\[ 0 < \delta < \min\{a_0, b_0/M\} \quad (\text{Teorema de Peano - unicidad global}) \]
En cualquier caso, el intervalo de existencia de la solución puede extenderse algo más allá de $[x_0 - \delta, x_0 + \delta]$. Notemos que
\[ (x_0 + \delta, \varphi(x_0 + \delta)) \in \Omega \]
Es admisible, entonces, plantear el siguiente problema de Cauchy:
\[ (\tilde{PC}) \begin{cases} y'(x) = f(x, y(x)) \\ y(x_0 + \delta) = \varphi(x_0 + \delta) \end{cases} \]
De igual forma, existirá $\delta_1 > 0$ tal que $(\tilde{PC})$ admite una única solución $\tilde{\varphi}$ en $[x_0 + \delta - \delta_1, x_0 + \delta + \delta_1]$. Gracias al Teorema de unicidad global, $\tilde{\varphi}$ se puede considerar como una prolongación de $\varphi$ hasta el punto $x_0 + \delta + \delta_1$.

\begin{definition}
  Dada $(I, \varphi) \in S(x_0, y_0)$, decimos que es:
  \begin{enumerate}
    \item Solución prolongable por la derecha si existe $(\tilde{I}, \tilde{\varphi}) \in S(x_0, y_0)$ tal que $I \subseteq \tilde{I}, I \neq \tilde{I}$ y $\sup(I) \in \text{int}(\tilde{I})$. En tal caso $\tilde{\varphi}|_{I} \equiv \tilde{\varphi}$.

    \item Solución prolongable por la izquierda si existe $(\tilde{I}, \tilde{\varphi}) \in S(x_0, y_0)$ tal que $I \subseteq \tilde{I}, I \neq \tilde{I}$ y $\inf(I) \in \text{int}(\tilde{I})$.

    \item Solución prolongable cuando es prolongable por la derecha, o por la izquierda, o por ambos extremos.

    \item Solución maximal cuando no es prolongable.
  \end{enumerate}
\end{definition}

\begin{theorem}
  Sea $\Omega \subseteq \mathbb{R}^{n+1}$ un conjunto abierto y $f \in C(\Omega, \mathbb{R}^n) \cap \text{Lip}_{\text{loc}}(y, \Omega)$. Para cada $(x_0, y_0) \in \Omega$ existe una única solución maximal del problema
  \[ (PC) \begin{cases} y'(x) = f(x, y(x)) \\ y(x_0) = y_0 \end{cases} \]
  y el intervalo $I \subseteq \mathbb{R}$ donde está definida la solución es abierto.
\end{theorem}

\begin{proof} (existencia, unicidad, estructura de una solución maximal)
  \begin{enumerate}
      \item[a)] En caso de existir una solución maximal, su dominio de definición es un intervalo abierto.

      \item[b)] Existe, a lo más, una solución maximal. Supongamos que existen dos soluciones maximales $(I_1, \varphi_1)$ e $(I_2, \varphi_2)$, con $x_0 \in I_1 \cap I_2$. Podemos definir $I = I_1 \cup I_2$ (que es un intervalo que contiene a $x_0$) y la función
      \[ \varphi(x) = \begin{cases} \varphi_1(x), & x \in I_1 \\ \varphi_2(x), & x \in I_2 \end{cases} \]
      Notar que $\varphi$ está bien definida gracias al Teorema de Unicidad global. Dado que $(I_1, \varphi_1)$ e $(I_2, \varphi_2)$ son soluciones maximales, entonces necesariamente debe ocurrir que $I_1 = I_2$, y por ende, $\varphi_1 \equiv \varphi_2$.

      \item[c)] Existencia de una solución maximal. Definamos
      \[ I(x_0, y_0) \coloneqq \bigcup_{(I, \varphi) \in S(x_0, y_0), I \text{ abierto}} I. \]
      Sabemos que $I(x_0, y_0) \neq \emptyset$ y que $I(x_0, y_0)$ es un intervalo, pues $x_0 \in I$ para cada $(I, \varphi) \in S(x_0, y_0)$. Denotemos por
      \[ \alpha = \inf I(x_0, y_0), \quad \beta = \sup I(x_0, y_0). \]
      Sabemos que $\alpha < x_0 < \beta$, pues al menos existe una solución en $([x_0 - \delta, x_0 + \delta], \varphi) \in S(x_0, y_0)$. Definimos ahora la función:
      \[ \tilde{\varphi} : I(x_0, y_0) \rightarrow \mathbb{R}^n \]
      del siguiente modo:
    
      Para cada $x \in I(x_0, y_0)$ existe, al menos, un intervalo abierto $I \subseteq \mathbb{R}$ $($con $x \in I)$ y una correspondiente solución local $\varphi \in C^1(I;\R^n)$ de $(PC)$ en $I$. Ponemos $\tilde{\varphi}(x) = \varphi(x)$. Por unicidad global sabemos que no hay ambigüedad en la definición de $\tilde{\varphi}$. Además
      \begin{itemize}
          \item $\tilde{\varphi}(x_0) = \varphi(x_0) = y_0;$
          \item $\tilde{\varphi}$ es de clase $C^1$ en $x$;
          \item $\tilde{\varphi}'(x) = \varphi(x) = f(x, \tilde{\varphi}(x))$.
      \end{itemize}
  \end{enumerate}
\end{proof}

\begin{note}
  Se denotará por $\varphi(\hspace{1mm} \cdot\hspace{1mm}; x_0, y_0)$ a la solución maximal del (PC), y a $I(x_0, y_0) \subseteq \mathbb{R}$ su intervalo (abierto) de definición.
\end{note}

\begin{theorem} 
  Sean $T_1 < T_2$ números reales y $\Omega \coloneqq (T_1, T_2) \times \mathbb{R}^n$. Supongamos que $f \in C(\bar{\Omega}, \mathbb{R}^n) \cap \text{Lip}_{\text{loc}}(y, \Omega)$ y que existen dos constantes $k_1, k_2 \in \mathbb{R}$ tales que
  \[ |f(x,y)| \leq k_1 + k_2|y| \quad \forall (x,y) \in \bar{\Omega} \]
  Entonces, dado cualquier $(x_0, y_0) \in \Omega$, existe una única solución del (PC) en $[T_1, T_2]$.
\end{theorem}

\begin{proof}
  Dado $(x_0, y_0) \in \Omega$, existe $a_0 > 0$ tal que
  \[ [x_0 - a_0, x_0 + a_0] \subseteq [T_1, T_2] \]
  Sea $b_0 \coloneqq k_1 + k_2|y_0|$, en modo tal que
  \[ R \coloneqq [x_0 - a_0, x_0 + a_0] \times \overline{B(y_0, b_0)} \subseteq \overline{\Omega} \]
  Notemos, por ende, que
  \[ \max_{(x,y) \in R} |f(x,y)| \leq \max_{y \in {B}(y_0, b_0)} (k_1 + k_2|y|) \leq k_1 + k_2b_0 + k_2|y_0| = b_0(1+k_2) \]
  Es decir, $M \leq b_0(1+k_2)$ y así,
  \[ \min\{a_0, b_0/M\} \geq \min\{a_0, 1/(1+k_2)\} > 0. \]
  y esta cantidad no depende de $(x_0, y_0)$. Por ende, repitiendo la demostración del Teorema de Peano, se puede seleccionar $\delta > 0$ tal que
  \[ 0 < \delta \leq \min\{a_0, 1/(1+k_2)\} \]
  y garantizar existencia de la solución de (PC) en $[x_0 - \delta, x_0 + \delta]$.
\end{proof}

\begin{note}
  La condición
  \[ (\exists A, B > 0) \quad |f(x,y)| \leq A + B|y| \quad \forall (x,y) \in \bar{\Omega}, \]
  expresa un crecimiento lineal de $f$ en $y$, uniformemente respecto a $x$. Tal comportamiento se puede garantizar en los siguientes casos:
  \begin{enumerate}
      \item[i)] $f \in L^\infty(\Omega)$ ($f$ acotada en $\Omega$).
      \item[ii)] $f(\cdot,0)$ es acotada y $f \in \text{Lip}(y, \bar{\Omega})$. En efecto, sabemos que existe $L > 0$ tal que
      \[ |f(x,y_1) - f(x,y_2)| \leq L|y_1 - y_2| \quad \forall (x,y_1), (x,y_2) \in \tilde{\Omega} \]
      Luego, para $(x,y) \in \tilde{\Omega}$ se tiene que
      \[ |f(x,y)| \leq |f(x,y) - f(x,0)| + |f(x,0)| \leq L|y| + \max_{x \in [T_1, T_2]} |f(x,0)| \]
      \item[iii)] $f(\cdot, 0)$ es acotada, $f \in \text{Lip}_{\text{loc}}(y;\tilde{\Omega})$ y sus derivadas parciales $\frac{\partial f}{\partial y_i}$, $i = \{1, \dots, n\}$ existen, son continuas y acotadas en $\bar{\Omega}$.
  \end{enumerate}
\end{note}
\end{document}
