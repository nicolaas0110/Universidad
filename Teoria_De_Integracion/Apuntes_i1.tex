\documentclass[11pt]{article}
\usepackage{indentfirst} 		 % First line paragraph indentation
\usepackage{etoolbox} 			% Used for being able to utilise if statements in the language detection.
%\setlength{\parskip}{\baselineskip}	% With this line, it is not needed the use of \\ at the end of each paragraph for...
					% ...spacing purposes, but the spacing can be extrange sometimes


% ========================= VARIABLES TO MODIFY =================================

\def\LANGUAGE{ES} %EN, ES		%Language in which the document is going to be written.

\def\reportTile{Apuntes}			%Document's title
\def\subject{Teoria De Integración}			%Subject

\def\writter{Nicolas Muñoz}		%Author(s)


\def\leftUpperHeader{\subject}		%\subject
\def\rightUpperHeader{\leftmark}	%\leftmark for current section


% Do NOT modify anything from here until the beginning of the document.
%============================== DATA PROCESSING ============================
\ifdefstring{\LANGUAGE}{EN}
{

\def\course{Licenciatura en Matemática}
\def\university{Pontificia Universidad Católica - Chile}
\def\pageCounterName{PAGE} 		% In the footer will be shown this... 
\def\pageSeparator{OUT OF} 		% ...text in addition of the page number.

}
{
\usepackage[utf8]{inputenc}		%Tildes en caso de no usar arch
\usepackage[spanish, es-tabla]{babel} 	%La opción es-tabla, hace que por defecto las tablas se llamen "Tabla", en vez de "Cuadro".
\def\course{Licenciatura en Matemática}
\def\university{Pontificia Universidad Católica - Chile}
\def\pageCounterName{PÁGINA}		%En el pie de página aparece este contenido más el número de página
\def\pageSeparator{DE} 			%DE, OUT OF

}


%===================== USEFUL VARIABLES  =============================
\def\imageSize{0.6}

%===================== PACKAGES TO USE  =============================
\usepackage{amsmath} 			% Allows me the use of matrix in math mode
\setcounter{MaxMatrixCols}{20} 		% Increases the maximum number of columns in matrix from 10 to 20.
\usepackage{pdfpages} 			% Attach pdfs using \includepdf[pages=initial-final]{path.pdf}
\usepackage{lastpage}			% Used to reference the last page and include it in the footer.
\usepackage{xcolor}			% Change color of text and so on. \textcolor{color}{text} is the one I use the most.
\usepackage{placeins} 			% Allows the use of the command \FloatBarrier
\usepackage{setspace}			% Allows the use of the spacing environment to change the space between lines.
\usepackage{graphicx}			% Figures addition
\usepackage{geometry}			% Changes the geometry of the pages
\usepackage[small, bf]{caption}		% Decreases the size of captions and turns them bold text.
\usepackage{subcaption}			% Include subfigures.
\usepackage{hyperref}			% Allows clickable references.
\hypersetup{colorlinks=true, allcolors=black} 	% Colors of links
\usepackage{pdflscape}			% Allows the use of the environment lscape for landscape
\usepackage{amssymb}
\usepackage{amsthm}
\usepackage{fancyhdr}			% Modify header and footer
\usepackage{bm}				% Allows to use bold text in math mode with \bm
\usepackage[makeroom]{cancel}		% Allows the use of \cancelto{}{} to cross out equations
\usepackage{titlesec}



\theoremstyle{definition} % hace que el cuerpo sea upright (no itálico)
\newtheorem{theorem}{Teorema}[section]
\newtheorem{lemma}[theorem]{Lema}
\newtheorem{proposition}[theorem]{Proposición}
\newtheorem{corollary}[theorem]{Corolario}
\newtheorem{definition}[theorem]{Definición}
\newtheorem{example}[theorem]{Ejemplo}
\newtheorem{remark}[theorem]{Observación}
\newtheorem{note}[theorem]{Nota}
\newcommand{\R}{\mathbb{R}}
\newcommand{\N}{\mathbb{N}}
\newcommand{\Q}{\mathbb{Q}}
\newcommand{\Z}{\mathbb{Z}}


%========== CHANGE TITLE FORMAT  ======================
%\titleformat{\section}
%{\bfseries}	% format
%{\thesection}	% label
%{0.3cm}		% separation between label and body
%{}		% code preceding title body
%[]		% code following title body


%========== HEADER AND FOOTER CONFIGURATION ======================
\pagestyle{fancy}
% HEADER[EVEN PAGES]{ODD PAGES}
% Left header
\lhead[
	\scriptsize{\MakeUppercase{\leftUpperHeader}} % Even pages
]
{
	\scriptsize{\MakeUppercase{\leftUpperHeader}} % Odd pages
}

% Central header
\chead[]{}

% Right header
\rhead[
\scriptsize{\rightUpperHeader} % Even pages
]
{
\scriptsize{\rightUpperHeader} % Odd pages
}

\renewcommand{\headrulewidth}{0.8pt}	% Width of the header rule

% FOOTER[EVEN PAGES]{ODD PAGES}
% Left footer
\lfoot[]{}

% Central footer
\cfoot[
\tiny{\pageCounterName\space\thepage\space \pageSeparator\space\pageref{LastPage}} % Even pages
]
{
\tiny{\pageCounterName\space\thepage\space \pageSeparator\space\pageref{LastPage}} % Odd pages
}

% Right footer
\rfoot[]{}
\renewcommand{\footrulewidth}{0.8pt} %Width of the footer rule


%================================== USER OWN FUNCTIONS ===============================================
\newcommand{\PDEA}[1]{\cdot 10^{#1}} % Función para escribir más rápidamente multiplicaciones por potencias de 10. PDEA = Por Diez Elevado A

\usepackage{xargs}	%Permite manejar argumentos opcionales en los comandos creados por el usuario

%Incluir imágenes, la sintaxis es la siguiente:
%\IncludeImage{ruta}[escalado][pie figura][etiqueta], los corchetes son opcionales
\newcommandx*{\IncludeImage}[4][2=1, 3=, 4=]{
	%#1 es la ruta a la imagen
	%#2 es el escalado
	%#3 es el pie de figura
	%#4 es la etiqueta
	
	\begin{figure}[h!]
		\centering
		\includegraphics[width=#2\linewidth]{#1}
		%Si se especifica pie de figura...
		\ifblank{#3}{}{
			\caption{#3}	
		}
		%Si se especifica etiqueta...
		\ifblank{#4}{}{
			\label{#4}	
		}
	\end{figure}
	\FloatBarrier
}



%================================ DOCUMENT BEGINNING  =============================================
\begin{document}
%***************** TITLE PAGE (DO NOT MODIFY) ******************************
\begin{titlepage}
	\newgeometry{top=2cm, bottom=3.5cm}

  %Logos Escuela y universidad
	\def\logoSize{0.2}
	\begin{figure}
  \hfill
	\includegraphics[width=\logoSize\linewidth]{Image/UC COLOR-01.png}
	\end{figure}


	%Espacio entre logos y título
	\vspace*{1.75cm}

	%Título con interlineado aumentado
	\begin{spacing}{2}
	\centering{
	\Huge{
		\textbf{\reportTile}
	}
	}
	\end{spacing}
	
	%Línea horizontal de la portada
	\hrule
	
	%Se baja al final de la hoja
	\vfill
	\begin{flushright}
	\LARGE{\writter}

	\vspace{2cm}
	
	\Large
	\subject\\
	\course\\
	\university
	
	\vspace{1cm}
	
	\today
	\end{flushright}
	
\end{titlepage}
	
\newgeometry{top=3cm, bottom=3cm}

\tableofcontents
\cleardoublepage

\section{Introducción a la Integración de Riemann}
\label{sec:riemann-intro}

\subsection{Particiones y Sumas de Riemann}

\begin{definition}
Una partición de un intervalo $[a,b] \subseteq \mathbb{R}$ es un subconjunto finito $\pi \subseteq [a,b]$ tal que $a,b \in \pi$. Denotaremos a las particiones como $\pi = \{x_0, \dots, x_n\}$ donde los puntos están ordenados, es decir $a=x_0 < x_1 < \dots < x_n = b$. Los intervalos $I_i = [x_{i-1}, x_i]$ para $i=1,\dots,n$ son llamados los intervalos de la partición. A veces identificaremos la partición con $(I_i)_{i=1,\dots,n}.$
\end{definition}

\begin{definition}
La norma de una partición $\pi$ se define como:
$$ ||\pi|| := \max_{i=1,\dots,n}(x_i - x_{i-1}) = \max_{I_i \in \pi} |I_i| $$
\end{definition}

\begin{definition}
Una partición marcada de $[a,b]$ es un par $\pi^* = (\pi, \epsilon)$, donde $\pi = \{x_0, \dots, x_n\}$ es una partición de $[a,b]$, y $\epsilon = \{x_1^*, \dots, x_n^*\}$ es una colección de puntos tal que $x_i^* \in I_i$ para cada $i=1,\dots,n$. La norma de una partición marcada se define como $||\pi^*|| = ||\pi||$.
\end{definition}

\begin{definition}[Suma de Riemann]
Sea $f:[a,b] \rightarrow \mathbb{R}$ acotada y $\pi^* = (\pi, \epsilon)$ una partición marcada. La suma de Riemann de $f$ asociada a $\pi^*$ se define como:
$$ S_R(f, \pi^*) = \sum_{i=1}^{n} f(x_i^*)(x_i - x_{i-1}) = \sum_{I_i \in \pi} f(x_i^*) |I_i| $$
\end{definition}

\begin{definition}[Integrabilidad de Riemann]
Dada $f:[a,b] \rightarrow \mathbb{R}$ acotada, decimos que es Riemann integrable si existe el límite:
$$ \lim_{||\pi^*|| \rightarrow 0} S_R(f, \pi^*) $$
Esto significa que $\exists L \in \mathbb{R}$ tal que para cualquier $\epsilon > 0$, existe $\delta = \delta(\epsilon) > 0$ tal que si $||\pi^*|| < \delta$, entonces $||S_R(f, \pi^*) - L|| < \epsilon$. Cuando este límite existe, lo llamamos la integral de Riemann de $f$ en $[a,b]$ y lo denotamos por $\int_{a}^{b} f(x)dx$.
\end{definition}

\subsection{Sumas de Darboux}
\label{sec:darboux}

\begin{definition}
Dadas $f:[a,b] \rightarrow \mathbb{R}$ acotada y $\pi = (I_i)_{i=1,\dots,n}$ una partición de $[a,b]$, definimos:
\begin{itemize}
    \item $m_{I_i} := \inf_{x \in I_i} f(x)$
    \item $M_{I_i} := \sup_{x \in I_i} f(x)$
    \item La suma inferior de Darboux: $\underline{S}(f; \pi) := \sum_{i=1}^{n} m_{I_i}(x_i - x_{i-1}) = \sum_{I_i \in \pi} m_{I_i} |I_i|$
    \item La suma superior de Darboux: $\overline{S}(f; \pi) := \sum_{i=1}^{n} M_{I_i}(x_i - x_{i-1}) = \sum_{I_i \in \pi} M_{I_i} |I_i|$
\end{itemize}
\end{definition}

\begin{remark}
Como $m_{I_i} \le f(x) \le M_{I_i}$ para todo $x \in I_i$, para cualquier partición marcada $\pi^* = (\pi, \epsilon)$, se tiene que:
$$ \underline{S}(f; \pi) \le S_R(f; \pi^*) \le \overline{S}(f; \pi) $$
\end{remark}

\begin{definition}[Refinamiento]
Una partición $\pi'$ de $[a,b]$ es un refinamiento de otra partición $\pi$ si $\pi \subset \pi'$. Equivalentemente, si para todo $J_i \in \pi'$ existe $I_i \in \pi$ tal que $J_i \subseteq I_i$.
\end{definition}

\begin{proposition}
Sea $f:[a,b] \rightarrow \mathbb{R}$ acotada. Entonces:
\begin{itemize}
    \item Si $\pi \subseteq \pi'$ son particiones de $[a,b]$, entonces $\underline{S}(f;\pi) \le \underline{S}(f;\pi')$ y $\overline{S}(f;\pi) \ge \overline{S}(f;\pi')$.
    \item Si $\pi_1, \pi_2$ son particiones de $[a,b]$ cualesquiera, entonces $\underline{S}(f;\pi_1) \le \overline{S}(f;\pi_2)$.
\end{itemize}
\end{proposition}

\subsection{Integrales de Darboux}
\label{sec:darboux-integrals}

\begin{definition}
Sea $f:[a,b] \rightarrow \mathbb{R}$ acotada. Definimos:
\begin{itemize}
    \item La integral superior (de Darboux) de $f$ como:
    $$ \overline{\int_{a}^{b}} f(x)dx := \inf_{\pi \text{ part. de } [a,b]} \overline{S}(f; \pi) $$
    \item La integral inferior (de Darboux) de $f$ como:
    $$ \underline{\int_{a}^{b}} f(x)dx := \sup_{\pi \text{ part. de } [a,b]} \underline{S}(f; \pi) $$
\end{itemize}
\end{definition}

\begin{theorem}
Sea $f:[a,b] \rightarrow \mathbb{R}$ acotada. Entonces:
$$ \underline{\int_{a}^{b}} f(x)dx = \lim_{||\pi|| \rightarrow 0} \underline{S}(f; \pi), \quad \overline{\int_{a}^{b}} f(x)dx = \lim_{||\pi|| \rightarrow 0} \overline{S}(f; \pi) $$
Equivalentemente, para cualquier sucesión $(\pi_n)_{n \in \mathbb{N}}$ de particiones de $[a,b]$ tal que $||\pi_n|| \rightarrow 0$ cuando $n \rightarrow \infty$, se tiene que:
$$ \underline{\int_{a}^{b}} f(x)dx = \lim_{n \rightarrow \infty} \underline{S}(f; \pi_n) \quad \text{y} \quad \overline{\int_{a}^{b}} f(x)dx = \lim_{n \rightarrow \infty} \overline{S}(f; \pi_n) $$
\end{theorem}

\begin{theorem}[Criterios de Integrabilidad]
Dada $f:[a,b] \rightarrow \mathbb{R}$ acotada, las siguientes afirmaciones son equivalentes:
\begin{enumerate}
    \item $f$ es integrable Darboux, es decir, $\underline{\int_{a}^{b}} f(x)dx = \overline{\int_{a}^{b}} f(x)dx$.
    \item $f$ es Riemann integrable.
    \item $\lim_{||\pi|| \rightarrow 0} (\overline{S}(f; \pi) - \underline{S}(f; \pi)) = 0$.
    \item Para cualquier sucesión $(\pi_n)_{n \in \mathbb{N}}$ de particiones de $[a,b]$ tal que $||\pi_n|| \rightarrow 0$, se tiene que $\lim_{n \rightarrow \infty} (\overline{S}(f; \pi_n) - \underline{S}(f; \pi_n)) = 0$.
    \item Existe una sucesión $(\pi_n)_{n \in \mathbb{N}}$ de particiones de $[a,b]$ tal que $\lim_{n \rightarrow \infty} (\overline{S}(f; \pi_n) - \underline{S}(f; \pi_n)) = 0$.
\end{enumerate}
\end{theorem}

\begin{remark}
Las integrales en el sentido de Darboux (1) y el de Riemann (2) coinciden.
\end{remark}

\begin{proposition}
\begin{itemize}
    \item Si $f:[a,b] \rightarrow \mathbb{R}$ es monótona, entonces es Riemann integrable.
    \item Si $f:[a,b] \rightarrow \mathbb{R}$ es continua, entonces es Riemann integrable.
\end{itemize}
\end{proposition}

\subsection{Medida de un conjunto}
\label{sec:measure}

\begin{definition}
Decimos que un conjunto $I \subseteq \overline{\mathbb{R}} := \mathbb{R} \cup \{-\infty, \infty\}$ es un intervalo si satisface que para todo $x,y \in I$, se tiene que $z \in I$ para todo $z$ tal que $\min\{x,y\} \le z \le \max\{x,y\}$.
\end{definition}

\begin{definition}
La medida de un intervalo $I \subseteq \overline{\mathbb{R}}$ se define como $|I| := \sup I - \inf I$. Se define $|\emptyset|:=0$ y $|x|:=0$ para un punto.
\end{definition}

\begin{property}
Si $I \subseteq J$ son intervalos, entonces $|I| \le |J|$.
\end{property}

\begin{definition}
Un conjunto $E \subseteq \mathbb{R}^d$ se dice de medida nula si, dado $\epsilon > 0$, existe una sucesión de intervalos $(I_n)_{n \in \mathbb{N}}$ de $\mathbb{R}^d$ tal que $E \subseteq \bigcup_{n \in \mathbb{N}} I_n$ y $\sum_{n \in \mathbb{N}} |I_n| < \epsilon$.
\end{definition}

\begin{theorem}
Sea $f:[a,b] \rightarrow \mathbb{R}$ acotada. Entonces, $f$ es Riemann integrable si y sólo si su conjunto de discontinuidades tiene medida nula.
\end{theorem}

\subsection{Limitaciones de la Integral de Riemann}
\label{sec:limitations}
La integral de Riemann tiene algunas limitaciones:
\begin{enumerate}
    \item Solo está definida para funciones acotadas y en intervalos $[a,b]$ acotados. Las integrales impropias resuelven parcialmente este problema.
    \item La convergencia puntual no siempre garantiza la intercambiabilidad del límite y la integral. Es decir, $f_n \rightarrow f$ puntualmente no implica que $\lim \int f_n = \int \lim f_n$. Ejemplos como $f_n(x) = n\chi_{(0, 1/n]}$ en $[0,1]$ muestran esta limitación.
\end{enumerate}

\begin{theorem}
Si $(f_n)_{n \in \mathbb{N}} \subseteq R([a,b])$ y $f_n \rightarrow f$ uniformemente en $[a,b]$, entonces $f \in R([a,b])$ y $\lim_{n \rightarrow \infty} \int_{a}^{b} f_n = \int_{a}^{b} f$.
\end{theorem}

\subsection{Teorema Fundamental del Cálculo}
\label{sec:ftc}

\begin{theorem}[Teorema Fundamental del Cálculo]
Si $f \in R([a,b])$ es continua en $x_0 \in [a,b]$, entonces $F(x) := \int_{a}^{x} f(t)dt$ es derivable en $x_0$ y $F'(x_0) = f(x_0)$. En particular, $F$ es derivable en $x$ y $F'(x) = f(x)$ para todo $x$ salvo un conjunto de medida nula.
\end{theorem}

\begin{note}
Este "casi" no puede removerse. Hay contraejemplos notables:
\begin{itemize}
    \item \textbf{Teorema de Hankel (1871):} Existe $f \in R([a,b])$ tal que $F(x) = \int_{a}^{x} f(t)dt$ no es derivable para ningún punto en un subconjunto denso de $[a,b]$.
    \item \textbf{Teorema de Volterra (1881):} Existe una función $f:[a,b] \rightarrow \mathbb{R}$ que es derivable en $[a,b]$ y su derivada $f'$ es acotada en $[a,b]$, pero $f' \notin R([a,b])$.
\end{itemize}
\end{note}

\section{Extendiendo la Integral de Riemann}
\label{sec:extending-riemann}

Una manera de extender el concepto de la integral es a través de funciones escalonadas.

\begin{definition}[Función Escalonada]
Una función $\phi:[a,b] \rightarrow \mathbb{R}$ se dice escalonada si existe una partición $\pi = \{x_0, \dots, x_n\}$ de $[a,b]$ y constantes $c_1, \dots, c_n \in \mathbb{R}$ tales que $\phi|_{(x_{i-1}, x_i)} \equiv c_i$ para todo $i=1,\dots,n$.
\end{definition}

Cualquier función escalonada se puede escribir como una combinación lineal de funciones características de intervalos. La integral de una función escalonada se define como:
$$ \int_{a}^{b} \phi(x)dx = \sum_{i=1}^{n} c_i |I_i| $$

\subsection{La Función Longitud}
\label{sec:length-function}

Sea $\mathcal{I}$ la colección de todos los intervalos en $\mathbb{R}$. La función longitud $\lambda:\mathcal{I} \rightarrow [0, \infty]$ se define como $\lambda(I) := |I|$.

\begin{property}
La función longitud $\lambda$ tiene las siguientes propiedades:
\begin{itemize}
    \item $\lambda(\emptyset) = 0$.
    \item \textbf{Monotonía:} Si $I_1, I_2 \in \mathcal{I}$ y $I_1 \subseteq I_2$, entonces $\lambda(I_1) \le \lambda(I_2)$.
    \item \textbf{Aditividad Finita:} Si $I \in \mathcal{I}$ tal que $I = \cup_{i=1}^{n} J_i$ con $J_i \in \mathcal{I}$ disjuntos, entonces $\lambda(I) = \sum_{i=1}^{n} \lambda(J_i)$.
    \item \textbf{Aditividad Contable ($\sigma$-aditividad):} Si $I \in \mathcal{I}$ es tal que $I = \cup_{i=1}^{\infty} I_i$ con $(I_i)_{i \in \mathbb{N}} \subseteq \mathcal{I}$ disjuntos, entonces $\lambda(I) = \sum_{i=1}^{\infty} \lambda(I_i)$.
    \item \textbf{$\sigma$-subaditividad:} Si $I \in \mathcal{I}$ verifica $I \subseteq \cup_{i=1}^{\infty} I_i$, donde $(I_i)_{i \in \mathbb{N}}$ son intervalos (no necesariamente disjuntos), entonces $\lambda(I) \le \sum_{i=1}^{\infty} \lambda(I_i)$.
    \item \textbf{Invarianza por traslaciones:} $\lambda(I+x) = \lambda(I)$ para todo $x \in \mathbb{R}$.
    \item $\lambda(\{x\}) = 0$ para todo $x \in \mathbb{R}$.
\end{itemize}

Nos gustaría extender $\lambda$ a una clase más grande que $\mathcal{I}$. Más precisamente, nos gustaría definir una aplicación $m : \mathcal{M} \to [0,\infty]$, donde $\mathcal{M}$ es una coleccción de subconjuntos de $\R$ tal que $\mathcal{I} \subseteq \mathcal{M}$, de manera tal que, dado $E \in \mathcal{M},\ m(E)$ represente la "longitud" de $E$. Idealmente, nos gustaría que $m$ cumpla lo siguiente:

	\begin{enumerate}
		\item $\mathcal{M} = \mathcal{P}(\R)$;

		\item Si $I \in \mathcal{I}$, entonces $m(I) = |I|$;

		\item $m$ es $\sigma$-aditiva ($E,\ (E_n)_{n\in\N}\in \mathcal{M},\ E = \displaystyle\bigcupd_{n=1}^{\infty} E_n \implies m(E) = \sum_{n=1}^{\infty} m(E_n)$);
	\end{enumerate}

	\begin{ex}
		$(1)+(2)+(3) \implies m$ es monóton, $\sigma$-subaditiva y finitamente aditiva. 
	\end{ex}

	\begin{enumerate}
		\item[4] Si $E \in \mathcal{M}$, entonces $E + x \in \mathcal{M}$ y $m(E + x) = m(E)\ \forall x \in \R$.
	\end{enumerate}

	\noindent El problema es que, si asumimos el Axioma de Elección, uno puede mostrar que no existe una tal $m$ que cumpla $(1)-(2)-(3)-(4)$ y, de hecho, no se sabe si existe $m$ que cumpla $(1)-(2)-(3)$. (Si asumimos la hipótesis del continuo, entonces no existe $m$ que cumpla $(1)-(2)-(3)$). \\

	\noindent Luego, para construir $m$ debemos debilitar alguna de las propiedades:
	\begin{itemize}
		\item Si debilitamos $(1) \implies$ TEORÍA DE LA MEDIDA;
		
		\item Si debilitamos $(3)$ pidiento solo (hay dos opciones):

		\begin{itemize}
			\item[$\rightarrow$] aditividad finita $ \implies$ "medidas finitamente aditivas";
		
			\item[$\rightarrow$] $\sigma$-subaditividad $\implies$ "medidas exteriores".
		\end{itemize}
	\end{itemize}

	\noindent Vamos a optar por debilitar $(1)$. \\\\
	\noindent Una manera de extender $\lambda$ es la siguiente:
	\begin{enumerate}
		\item[i.] Si $E = \displaystyle\bigcupd_{i=1}^{n} I_i$ entonces defninimos $\lambda(E) \coloneq \sum_{i=1}^{n} \lambda (I_i)$;

		\item[ii.] Si $E = \displaystyle\bigcupd_{i=1}^{\infty} I_i$ entonces definimos $\lambda(E) \coloneq \sum_{i=1}^{\infty} \lambda (I_i)$;

		\item[iii.] La fórmula anterior nos permite definir $\lambda (6)$ para todo $6$ abierto en $\R$;

		\item[iv.] Para conjuntos mas generales, "aproximar" por abiertos. 
	\end{enumerate}

	\begin{definition}[premedida]
		Sea $X$ un conjunto no vacío y $\mathscr{C}$ una colección de subconjuntos de $X$ tal que $\varnothing \in \mathscr{C}$. Diremos que una aplicación $\mathcal{T} : \mathscr{C} \to [0,\infty]$ es una premedida si $\mathcal{T} (\varnothing)=0$.
	\end{definition}

	\begin{remark}
		El conjunto no vacío $X$ será llamado un espacio y la colección $\mathscr{C}$ será llamada una clase (de subconjuntos de $X$).
	\end{remark}

	\noindent Intuitivamente, $\mathscr{C}$ representa la colección de subconjuntos cuyo "tamaño" sabemos medir y $\mathcal{T}$ nos da su medida.

	\begin{eg}~
		\begin{enumerate}
			\item \textbf{Premedida de Lebesgue:} $\mathscr{C} \coloneq \mathcal{I} \coloneq \{ I \subseteq \R \ : \ I \text{ intervalo}\}, \ \mathcal{T}(I) \coloneq |I|$.

			\item \textbf{Premedidas de Lebesgue-Stieltjes:} Sea $F:\R \to \R$ monótona creciente y continua a derecha ($\lim_{x \to x_0}^{+} F(x) = F(x_0)$). Una función tal se dice una función de Lebesgue-Stieltjes. 
		\end{enumerate}
	\end{eg}

	\noindent Observemos que, por monotonía, existen límites \[ \left\{ \begin{aligned}
		F(\infty) & \coloneq \lim_{x \to \infty} F(x) \\ 
		F(-\infty) & \coloneq \lim_{x \to -\infty} F(x) 
	\end{aligned} \right\} \in \R \]

	\noindent Sea además la clase $\widetilde{\mathcal{I}}$ de intervalos de $\R$ dada por
	\begin{align*}
		\widetilde{\mathcal{I}} & \coloneq \{ I(a,b) \ : \ \} \text{ donde } I(a,b) \coloneq (a,b] \cap \R \\
		& = \{ (a,b] \ : \ -\infty \leq a \leq b \} \cup \{ (a,\infty) \ : \ -\infty \leq a < \infty \}. 
	.\end{align*}

	\noindent Definimos la premedida $\mathcal{T}_F$ de Lebesgue-Stieltjes asociada a $F$ como la aplicación $\mathcal{T}_F: \widetilde{\mathcal{I}} \to [0, \infty]$, dada por
	\[ \mathcal{T}_F (I(a,b)) = F(b) - F(a) .\]

	\begin{note}
		Observar que si $F(x)=x$ entonces $\mathcal{T}_F$ es la premedida de Lebesgue (sobre $\widetilde{\mathcal{I}}$.
	\end{note}

	\begin{enumerate}
		\item[3.] \textbf{Premedidas de Probabilidad:} Si $F$ es una función de L-S tal que $F(\infty) = 1$ y $F(-\infty) = 0$, decimos que $F$ es una función de distribución (acumulada). En tal caso, la premedida $\mathcal{T}_F$ se conoce como premedida de probabilidad o predistribución (en $\R$).
	\end{enumerate}

	\begin{remark}
		$\mathcal{T}_F (\R) = \mathcal{T}_F (I(-\infty,\infty)) = F(\infty) - F(-\infty) = 1 - 0 = 1$.
	\end{remark}

	\begin{enumerate}
		\item[4.] \textbf{Premedida...}
	\end{enumerate}

  \begin{definition}[semiálgebra]
		Sea $X$ un espacio y $\mathscr{C}$ una clase de subconjuntos de $X$. Decimos que $\mathscr{C}$ es una semiálgebra (de subconjuntos de $X$) si cumple:
		\begin{enumerate}
			\item $\varnothing \in \mathscr{C}$;

			\item ($\mathscr{C}$ es cerrada por intesecciones finitas) $A,B\in\mathscr{C} \implies A \cap B \in \mathscr{C}$;
			
			\item Si $A \in \mathscr{C}$, existen $C_1,\dots,C_n \in \mathscr{C}$ disjuntos tal que $A^c = \displaystyle\bigcupd_{i=1}^{n} C_i$.
		\end{enumerate}
	\end{definition}
	
	\begin{eg}~
		\begin{enumerate}
			\item La clase $\mathcal{I}_d$ de intervalos en $\R^d$ es una semiálgebra.

			\item La clase $\widetilde{\mathcal{I}} \coloneq \{ (a,b] \cap \R \ : \ -\infty \leq a \leq b \leq \infty \}$ es una semiálgebra.

			\item Si $X$ e $Y$ son espacios y $\mathscr{C}_X, \mathscr{C}_Y$ son semiálgebras en $X$ e $Y$ respectivamente, entonces
			\[ \mathscr{C}_X \times \mathscr{C}_Y \coloneq \{ F \times G \ : \ F \in \mathscr{C}_X,\ G \in \mathscr{C}_Y \} \]
			es una semiálgebra en $X \times Y$, llamada "semiálgebra producto".
		\end{enumerate}
	\end{eg}

	\begin{definition}[álgebra]
		Sean $X$ un espacio y $\mathscr{A}$ una clase de subconjuntos de $X$. Decimos que $\mathscr{A}$ es un álgebra (de subconjuntos de $X$) si cumple que:
		\begin{enumerate}
			\item[(i)] $\varnothing \in \mathscr{A}$;
			
			\item[(ii)] $\mathscr{A}$ es cerrado por intersecciones finitas;

			\item[(iii)] ($\mathscr{A}$ es cerrada por complementos) $A \in \mathscr{A} \implies A^c \in \mathscr{A}$. 
		\end{enumerate}	
			\noindent Equivalentemente, en presencia de (iii), (ii) se puede reemplazar por:	
		\begin{enumerate}
			\item[(ii')] ($\mathscr{A}$ es cerrada por uniones finitas) $A,B\in \mathscr{A} \implies A \cup B \in \mathscr{A}$. (\textbf{Dem:} Ejercicio!)
		\end{enumerate}
	\end{definition}

	\begin{eg}~
		\begin{enumerate}
			\item $X$ espacio, $\mathscr{A}_1 \coloneq \{\varnothing, X\},\ \mathscr{A}_2 \coloneq \mathcal{P}(X)$ son álgebras (donde $\mathscr{A}$ es llamada el álgebra trivial);

			\item Sea $\mathscr{S}$ una semiálgebra de subconjuntos de un espacio $X$. Entonces 
			\[ \mathscr{A} \coloneq \{ E \subseteq X \ : \ \exists S_1,\dots,S_n \in \mathscr{S} \text{ disjuntos tal que } E = \displaystyle\bigcupd_{i=1}^{n} S_i \} \] 
			es un álgebra, llamada el álgebra generada por $\mathscr{S}$. Notemos que $\mathscr{A}(\mathscr{S}$ es el menor álgebra que contiene a $\mathscr{S}$:
			\begin{enumerate}
				\item[(i)] $\mathscr{A}(\mathscr{S})$ es un álgebra y $\mathscr{S} \subseteq \mathscr{A}(\mathscr{S})$;

				\item[(ii)] Si $\mathscr{A}'$ es un álgebra con $\mathscr{S} \subseteq \mathscr{A}'$ entonces $\mathscr{A}(\mathscr{S} \subseteq \mathscr{A}'$.
			\end{enumerate}
		\end{enumerate}
	\end{eg}

	\begin{note}
		Toda álgebr es una semiálgebra.
	\end{note}

	\begin{definition}[$\sigma$-álgebra]
		Una clase (no vacía) $\mathscr{M}$ de subconjuntos de un espacio $X$ se dice una $\sigma$-álgebra si cumple:
		\begin{enumerate}
			\item $\varnothing \in \mathscr{M}$;

			\item $E \in \mathscr{M} \implies E^c \in \mathscr{M}$;

			\item $(E_n)_{n\in\N} \subseteq \mathscr{M} \implies \displaystyle\bigcup_{n\in\N} E_n \in \mathscr{M}$.
		\end{enumerate}
		\noindent Llamamos al par $(X,\mathscr{M})$ un \underline{espacio medible} y a los elementos de $\mathscr{M}$, \underline{conjuntos medibles}.
	\end{definition}

	\begin{note}~
		\begin{enumerate}
			\item Todo $\sigma$-álgebra es un álgebra;

			\item Equivalentemente, en presencia de (1), (3) se puede reemplazar por
			\begin{enumerate}
				\item[(iii')] $(E_n)_{n\in\N} \subseteq \mathscr{M} \implies \displaystyle\bigcap_{n\in\N} E_n \in \mathscr{M}$.
			\end{enumerate}
		\end{enumerate}
	\end{note}

	\begin{eg}~
		\begin{enumerate}
			\item $\sigma$-álgebra $\implies$ álgebra $\implies$ semiálgebra (no valen las recíprocas);

			\item $\{\varnothing,X\},\mathcal{P}(X)$ son $\sigma$-álgebras;

			\item Si $(\mathscr{M}_{\gamma})_{\gamma \in \Gamma}$ son $\sigma$-álgebras, entonces
			\[ \displaystyle\bigcap_{\gamma\in\Gamma} \mathscr{M}_{\gamma} \coloneq \{ E \subseteq X \ : \ E \in \mathscr{M}_{\gamma},\ \forall \gamma \in \Gamma \} \]
			es una $\sigma$-álgebra.
			
			\item Si $\mathscr{M}$ es una clase de subconjuntos de $X$, entonces
			\[ \sigma(\mathscr{M}) \coloneq \displaystyle\bigcap_{\begin{aligned}
				\mathscr{M}& \ \sigma\text{-álgebra} \\
				& \mathscr{C}\subseteq\mathscr{M}
			\end{aligned}} \mathscr{M} \]
			es la $\sigma$-álgebra generada por $\mathscr{C}$. De hecho, $\sigma(\mathscr{M})$ es la menor $\sigma$-álgebra que contiene a $\mathscr{C}$:
			\begin{enumerate}
				\item $\sigma(\mathscr{C})$ es $\sigma$-álgebra y $\mathscr{C} \subseteq \sigma(\mathscr{C})$;

				\item Si $\mathscr{F}$ es $\sigma$-álgebra y $\mathscr{C} \subset \mathscr{F}$ entonces $\sigma(\mathscr{C}) \subseteq \mathscr{F}$.
			\end{enumerate}

			\item Si $(X,\mathscr{T})$ es un espacio topológico, $\sigma(\mathscr{T})$ se conoce como la \underline{$\sigma$-álgebra} \underline{de Borel}, y sus elementos se llaman \underline{Borelianos}. La notamos $\beta(X)\ (= \sigma(\mathscr{T}))$.
		\end{enumerate}
	\end{eg}
	
	\begin{eg}
		$\beta(\R)$ contiene a tods los abiertos, cerrados, intervalos, conjuntos de tipo $G_{\delta}$ y $F_{\sigma},\dots$ De hecho, $\beta(\R)=\sigma(\text{cerrados})=\sigma(\text{compactos})=\sigma(\mathcal{I})=\sigma(\widetilde{\mathcal{I}})$.
	\end{eg}

	\begin{definition}
		Sea $\mathscr{C}$ una clase (no vacía) de subconjuntos de $X$ y $\mu : \mathscr{C} \to [0,\infty]$ una función (la llamamos una \underline{función de conjuntos}). Diremos que:
		\begin{enumerate}
			\item[(i)] \textbf{$\mu$ es monótona} (en $\mathscr{M}$) si $A,B\in\mathscr{C},\ A \subseteq B \implies \mu(A)\leq\mu(B)$;

			\item[(ii)] \textbf{$\mu$ es finitamente aditiva} si $(A_i)_{i=1,\dots,n} \subseteq \mathscr{C}$ disjuntos $\implies \mu(\bigcupd_{i=1}^{n} A_i) = \sum_{i=1}^{n} \mu(A_i)$;

			\item[(iii)] \textbf{$\mu$ es $\sigma$-aditiva} si $(A_n)_{n\in\N} \subseteq \mathscr{C}$ disjuntos $\implies \mu (\bigcupd_{i=1}^{\infty} A_i ) = \sum_{i=1}^{\infty} \mu(A_i)$;

			\item[(iv)] \textbf{$\mu$ es $\sigma$-subaditiva} si $\mu(A)\leq \sum_{i=1}^{\infty}\mu(A_n)$, para todo $A \in \mathscr{C}$ y $(A_n)_{n\in\N}\subseteq\mathscr{C}$ tal que $A \subseteq \bigcup_{n\in\N} A_n$
		\end{enumerate}
	\end{definition}

  \begin{remark}
	Rana da una definición más débil de (4):
	\[ A \in \mathscr{C},\ A = \bigcup_{i=1}^{\infty} A_i,\ A_i \in \mathscr{C}\ \forall i \implies \mu(A) \leq \sum_{i=1}^{\infty}\mu(A_i) \]
	\noindent Ambas definiciones son equivalentes si $\mathscr{C}$ es una semiálgebra y $\mu$ es monótona (siempre será el caso para nosotros).
\end{remark}

\begin{definition}[premedida finita y $\sigma$-finita]
	Una premedida $\mathcal{T} : \mathscr{C} \to [0,\infty]$ se dice:
	\begin{enumerate}
		\item \textbf{finita} si $X \in \mathscr{C}$ y $\mathcal{T} < \infty$;

		\item \textbf{$\sigma$-finita} si existen $(C_n)_{n\in\N} \subseteq \mathscr{C}$ \underline{disjuntos} tales que $\bigcupd_{n=1}^{\infty} C_n = X$ y $\mathcal{T} (C_n) < \infty\ \forall n \in \N$.
	\end{enumerate}
\end{definition}

\begin{eg}~
	\begin{enumerate}
		\item finita $\implies \sigma$-finita;

		\item La función longitud $\lambda : \mathcal{I} \to [0,\infty]$ es $\sigma$-finita pero no finita;

		\item Si $F$ es una función de L-S, entonces $\mathcal{T}_F : \widetilde{\mathcal{I}} \to [0,\infty]$ es siempre $\sigma$-finita $(\mathcal{T}_F ((n,n+1]) = F(n+1) - F(n) < \infty\ \forall n \in \Z)$ y es finita si y sólo si $\mathcal{T}_F (\R) = \mathcal{T}_F ((-\infty,\infty]\cap\R) = F(\infty)-F(-\infty) < \infty$.
	\end{enumerate}
\end{eg}

\begin{definition}[medida]
	Sea $(X,\mathscr{M})$ es un espacio medible. Diremos que $\mu : \mathscr{M} \to [0,\infty]$ es una \underline{medida} (en $(X,\mathscr{M})$) si:
	\begin{enumerate}
		\item $\mu (\varnothing) = 0$;
		
		\item $\mu$ es $\sigma$-subaditiva en $\mathscr{M}\ \left( \mu \left(\bigcupd_{i=1}^{\infty} A_i \right) = \sum_{i=1}^{\infty} \mu(A_i) \right)$.
	\end{enumerate}
	\noindent Llamamos a la terna $(X,\mathscr{M},\mu)$ un \underline{epacio de medida}.
\end{definition}

\noindent \textbf{Objetivo.} Construir un espacio de medida $(\R,\mathscr{M},\mu)$ tal que $\mathcal{I} \subseteq \mathscr{M}$ y
\[ \begin{cases}
	\mu(I) = |I|\ \forall I \in \mathcal{I}, \\
	\mu(E+x) = \mu(E) \ \forall E \in \mathscr{M}.
\end{cases} \]

\begin{eg}[Espacios de Probabilidad]
	Si $(X,\mathscr{M},\mu)$ es un EdM tal que $\mu(X)=1,\ (X,\mathscr{M},\mu)$ recibe el nombre de \underline{espacios de probabilidad}.
\end{eg}

\begin{itemize}
	\item $X$ recibe el nombre de \underline{espacio muestral}, y se lo nota $\Omega$ (en lugar de $X$);

	\item $\mathscr{M}$ se suele notar como $\mathscr{F}$ (ó $\mathscr{Y}$). Sus elementos se dicen \underline{eventos};

	\item $\mu$ recibe el nombre de \underline{medida de probabilidad} ó \underline{distribución} y se la nota $\mathbb{P}$.
\end{itemize}

\noindent En probabilidad, típicamente se estudian $2$ tipos de distribuciones en $\R$ (o en $\R^d$).

\begin{enumerate}
	\item \textbf{Distribuciones discretas:} $\exists S \subseteq \R$ numerable y $(p_x)_{x \in S} \subseteq [0,1]$ tal que $\mathbb{P}(A) = \sum_{x \in A \cap S} p_x$.
	\begin{eg}
		Binomial, Geométrica, Poisson,...
	\end{eg}

	\item \textbf{Distribuciones (absolutamente) continuas:} $\exists f : \R \to \R_{\geq 0}$ "integrable" tal que $\mathbb{P}(A) = \int_{A} f(x) dx$.
	\begin{eg}
		Uniforme, Exponencial, Normal,...
	\end{eg}
\end{enumerate}

\noindent \textbf{Propiedades generales de una medida.} Si $\mu$ es una medida sobre $(X,\mathscr{M})$, entonces:
\begin{enumerate}
	\item $\mu$ es monótona (en $\mathscr{M}$);

	\item $\mu$ es $\sigma$-subaditiva;

	\item $\mu$ es \textbf{continua por debajo}: si $(A_n)_{n\in\N} \subseteq \mathscr{M}$ es \underline{creciente} $(A_n \subseteq A_{n+1}\ \forall n)$ entonces
	\[ \mu \left( \bigcup_{n\in\N} A_n \right) = \lim_{n \to \infty} \mu(A_n). \]

	\item $\mu$ es \textbf{continua por arriba}: si $(A_n)_{n\in\N} \subseteq \mathscr{M}$ es \underline{decreciente} $(A_{n+1} \subseteq A_n\ \forall n)$ y $\mu(A_{n_0})<\infty$ para algún $n_0\ (\implies \mu(A_n)<\infty\ \forall n\geq n_0)$, entonces
	\[ \mu \left( \bigcap_{n\in\N}A_n \right) = \lim_{n \to \infty} \mu(A_n). \]

	\noindent (\textbf{Cuidado!} (4) puede no valer si $\mu (A_n) = \infty \ \forall n \in \N$)
\end{enumerate}

\begin{definition}[premedida extendible y unívocamente extendible]
	Una premedida $\mathcal{T} : \mathscr{S} \to [0,\infty]$ definida sobre una semiálgebra de subconjunto de $X$, se dice:
	\begin{enumerate}
		\item \textbf{Extendible} si es
		\begin{enumerate}
			\item[(E1)] finitamente aditiva en $\mathscr{S}$;

			\item[(E2)] $\sigma$-subaditiva en $\mathscr{S}$.
		\end{enumerate}

		\item \textbf{Unívocamente extendible} si es extendible y se cumple
		\begin{enumerate}
			\item[(E3)] $\sigma$-finita
		\end{enumerate}
	\end{enumerate}
\end{definition}

\begin{remark}
	Los nombres de extendible y unívocamente extendible no se encontrarán en el Rana (los puso el profe).
\end{remark}

\begin{theorem}[Extensión de Carathéodory]
	Dados un espacio $X$ y una premedida $\mathcal{T}$ sobre una semiálgebra $\mathscr{S}$ de subconjuntos de $X$ tal que $\mathcal{T}$ es extendible, existe una extensión de $\mathcal{T}$ a una medida $\mu_{\mathcal{T}}$ definida sobre $\sigma(\mathscr{S})$ la $\sigma$-álgebra generada por $\mathscr{S}$. Más aún, si $\mathcal{T}$ es unívocamente extendible, entonces la extensión $\mu_{\mathcal{T}}$ a $\sigma(\mathscr{S})$ es \underline{única}. \\
	Por último, si $\mathcal{T}$ es unívocamente extendible, entonces se puede extender de manera única a una medida $\overline{\mu_{\mathcal{T}}}$ sobre la $\mu_{\mathcal{T}}$-completación de $\sigma(\mathscr{S})$, i.e. la $\sigma$-álgebra $\overline{\sigma(\mathscr{S})}$ dada por
	\[ \overline{\sigma(\mathscr{S})} \coloneq \{ B \cup N : B \in \sigma(\mathscr{S}), \exists \widetilde{N} \in \sigma(\mathscr{S}) \text{ con } N \subseteq \widetilde{N} \text{ y } \mu_{\mathcal{T}} (\widetilde{N}) = 0 \} \]
	\noindent mediante la fórmula $\overline{\mu_{\mathcal{T}}}(B\cap N) \coloneq \mu_{\mathcal{T}}(B)$.
\end{theorem}

\begin{remark}
	Si $\mathcal{T} : \mathscr{S} \to [0,\infty]$ es $\sigma$-aditiva en $\mathscr{S}$ y $\mathscr{S}$ es una semiálgebra, entonces $\mathcal{T}$ es extendible.
\end{remark}

\begin{remark}
	La extensión puede no ser única si $\mathcal{T}$ no es $\sigma$-finita.
\end{remark}
\begin{eg}
	$\widetilde{\mathcal{I}}_{\Q} \coloneq \widetilde{\mathcal{I}} \cap \Q = \{ (a,b] \cap \Q \ : \ -\infty \leq a \leq b \leq \infty \}$ 
\end{eg}

\begin{note}~
	\begin{itemize}
		\item $\widetilde{\mathcal{I}}_{\Q}$ es una semiálgebra;

		\item $\sigma (\widetilde{\mathcal{I}}_{\Q}) = \sigma (\widetilde{\mathcal{I}} \cap \Q) \stackrel{\text{Ej!}}{=} \sigma (\widetilde{\mathcal{I}}) \cap \Q = \beta(\R) \cap \Q = \mathcal{P}(\Q)$ (9.52)

	\item $\mathcal{T} : \widetilde{\mathcal{I}}_{\Q} \to [0,\infty]$, dada por $\mathcal{T}(A) \coloneq \begin{cases}
			0 \quad A = \varnothing \\
			\infty \quad A \neq \varnothing,\ A \in \widetilde{\mathcal{I}}_{\Q}
		\end{cases}$ (Observar que $\mathcal{T}$ no es $\sigma$-finita)

		\item Para cada $r > 0,\ \mu_r : \mathcal{P}(\Q) \to [0,\infty]$ dada por $\mu_r(A) \coloneq r(\# A)$ es una extensión de $\mathcal{T}$ (y es una medida)
	\end{itemize}
\end{note}

\begin{definition}[espacio completo y conjuntos $\mu$-nulos]
	Sea $(X,\mathscr{M},\mu)$ un EdM y definamos 
	\[ \mathscr{N}_{\mu} \coloneq \{ E \subset X \ : \ \exists N \in \mathscr{M} \text{ con } E \subseteq N \text{ y } \mu(N) = 0 \} \] 
	Los elementos de $\mathscr{N}_{\mu}$ se dicen \underline{conjuntos $\mu$-nulos}. Diremos que $(X,\mathscr{M},\mu)$ es \underline{completo} si $\mathscr{N}_{\mu} \subseteq \mathscr{M}$
\end{definition}

\begin{remark}
	$(X,\overline{\sigma(\mathscr{S})},\overline{\mu_{\delta}})$ es \underline{completo}. En efecto, $\mathscr{N}_{\overline{\mu_{\delta}}}$ corresponde al subconjunto de $\overline{\sigma(\mathscr{S})}$ que se obtiene tomando $B=\varnothing$.
\end{remark}

\begin{remark}
	Veremos más adelante que las siguientes premedidas son UE:

	\begin{itemize}
		\item[(i)] Premedidas de Lebesgue-Stieltjes (en particular, la función longitud $\lambda$ (sobre $\widetilde{\mathcal{I}}$) y las premedidas de probabilidad).

		\item[(ii)] Premedidas de Lebesgue en $\R^d$, con $d \in \N$.
	\end{itemize}
\end{remark}

\noindent En particular;

\begin{corollary}
	Para cada función $F$ de Lebesgue-Stieltjes, existe una $\sigma$-álgebra $\mathscr{M}_F$ sobre $\R$ y una única medida $\mu_F$ en $(\R,\mathscr{M}_F)$ tal que
	\[ \mu_F = (I(a,b)) = F(b) - F(a) \quad \forall -\infty \leq a \leq b \leq \infty \]
	Además, $\beta(\R) \subseteq \mathscr{M}_F$. Es decir, $\mu_F$ es una medida que extiende a $\mathcal{T}_F$, a todo $\mathscr{M}_F$ (y en particular, a todo $\beta(\R)$). Además, $(\R, \mathscr{M}_F, \mu_F)$ es un EdM completo. ($\mathscr{M}_F \coloneq \overline{\sigma(\widetilde{\mathcal{I}})^F},\ \mu_F \coloneq \overline{\mu_{\mathcal{T}_F}}$). La medida $\mu_F$ se conoce como \underline{medida de L-S asociada a $F$}. En particular, para cualquier función de distribución $F$, existe una única medida de probabilidad $\mathbb{P}_F$ en $(\R,\beta(\R))$ tal que 
	\[ \mathbb{P}_F(I(a,b)) = F(b) - F(a) \quad \forall -\infty \leq a \leq b \leq \infty \]
	(En la guía 3 veremos que $F \to \mathbb{P}_F$ es una biyección)
\end{corollary}

\begin{note}
	Los $\beta$ son los Borelianos y $I(a,b) = (a,b] \cap \R$. (super $F \to$ 10.26).
\end{note}

\begin{eg}[Importante!]
	\textbf{Medida de Lebesgue en $\R$.} Tomando $F = id$ en el Corolario anterior, obtenemos una $\sigma$-álgebra $\mathscr{L}(\R) \coloneq \mathscr{M}_{id}$ con $\beta(\R) \subseteq \mathscr{L}(\R)$ y una medida $\mu_{id}$ en $(\R,\mathscr{L}(\R))$ tal que $\mu_{id}(I(a,b)) = b-a \quad \forall -\infty \leq a \leq b \leq \infty$. En particular, de esto se deduce que $\mu_{id}(I) = |I|\quad \forall I \in \mathcal{I}$. Dicha medida recibe el nombre de \underline{medida de Lebesgue} (en $\R$), y los elementos de $\mathscr{L}(\R)$ se dicen \underline{conjuntos medibles Lebesgue}. Adoptaremos la notación $\mu_{id}(E) \coloneq \lambda(E) \coloneq |E|$. La medida $\mu_{id}$ \underline{es} la extensión de la noción de longitud que buscábamos y $\mathscr{L}(\R)$ son los conjuntos cuya "longitud" podremos medir. Además, los conjuntos de medida nula (de la guía 2), son \underline{exactamente} aquellos $A \in \mathscr{L}(\R)$ tal que $\mu_{id}(A) = 0$ (lo veremos más adelante!).
\end{eg}

\begin{eg}[Medida de Lebesgue en $\R^d$]
	Si $\mathcal{I}_d$ son los intervalos en $\R^d$ y definimos $\mathcal{T}:\mathcal{I}_d \to [0,\infty]$ como $\mathcal{T}(I)\coloneq|I|$, entonces $\mathcal{I}_d$ es una semiálgebra y $\mathcal{T}$ es una premedida $\sigma$-aditiva en $\mathcal{I}_d$ (lo veremos después). Por lo tanto, $\mathcal{T}$ se puede extender (de manera única, pues $\mathcal{T}$ es $\sigma$-finita) a una medida $\mu_{\delta}$ sobre la $\sigma$-álgebra $\mathscr{L}(\R^d) = \overline{\sigma(\mathcal{I}_d)^{\mathcal{T}}}$, llamada \underline{medida de Lebesgue en $\R^d$} y $\mathscr{L}(\R^d)$ es la clase de \underline{conjuntos medibles Lebesgue en $\R^d$}. Al igual que antes, dado $E \in \mathscr{L}(\R^d)$, notamos $|E| \coloneq \mu_{\mathcal{T}}(E)$.
\end{eg}

\end{document}
