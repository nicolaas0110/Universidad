\documentclass[11pt]{article}
\usepackage{indentfirst} 		 % First line paragraph indentation
\usepackage{etoolbox} 			% Used for being able to utilise if statements in the language detection.
%\setlength{\parskip}{\baselineskip}	% With this line, it is not needed the use of \\ at the end of each paragraph for...
					% ...spacing purposes, but the spacing can be extrange sometimes


% ========================= VARIABLES TO MODIFY =================================

\def\LANGUAGE{ES} %EN, ES		%Language in which the document is going to be written.

\def\reportTile{Tarea 1}			%Document's title
\def\subject{Álgebra Lineal}			%Subject

\def\writter{Alonso Muñoz}		%Author(s)


\def\leftUpperHeader{\subject}		%\subject
\def\rightUpperHeader{\leftmark}	%\leftmark for current section


% Do NOT modify anything from here until the beginning of the document.
%============================== DATA PROCESSING ============================
\ifdefstring{\LANGUAGE}{EN}
{

\def\course{Licenciatura en Matemática}
\def\university{Pontificia Universidad Católica - Chile}
\def\pageCounterName{PAGE} 		% In the footer will be shown this... 
\def\pageSeparator{OUT OF} 		% ...text in addition of the page number.

}
{
\usepackage[utf8]{inputenc}		%Tildes en caso de no usar arch
\usepackage[spanish, es-tabla]{babel} 	%La opción es-tabla, hace que por defecto las tablas se llamen "Tabla", en vez de "Cuadro".
\def\course{Licenciatura en Matemática}
\def\university{Pontificia Universidad Católica - Chile}
\def\pageCounterName{PÁGINA}		%En el pie de página aparece este contenido más el número de página
\def\pageSeparator{DE} 			%DE, OUT OF

}


%===================== USEFUL VARIABLES  =============================
\def\imageSize{0.6}

%===================== PACKAGES TO USE  ============================
\usepackage{amsmath} 			% Allows me the use of matrix in math mode
\setcounter{MaxMatrixCols}{20} 		% Increases the maximum number of columns in matrix from 10 to 20.
\usepackage{pdfpages} 			% Attach pdfs using \includepdf[pages=initial-final]{path.pdf}
\usepackage{lastpage}			% Used to reference the last page and include it in the footer.
\usepackage{xcolor}			% Change color of text and so on. \textcolor{color}{text} is the one I use the most.
\usepackage{placeins} 			% Allows the use of the command \FloatBarrier
\usepackage{setspace}			% Allows the use of the spacing environment to change the space between lines.
\usepackage{graphicx}			% Figures addition
\usepackage{geometry}			% Changes the geometry of the pages
\usepackage[small, bf]{caption}		% Decreases the size of captions and turns them bold text.
\usepackage{subcaption}			% Include subfigures.
\usepackage{hyperref}			% Allows clickable references.
\hypersetup{colorlinks=true, allcolors=black} 	% Colors of links
\usepackage{pdflscape}			% Allows the use of the environment lscape for landscape
\usepackage{amssymb}
\usepackage{amsthm}
\usepackage{fancyhdr}			% Modify header and footer
\usepackage{bm}				% Allows to use bold text in math mode with \bm
\usepackage[makeroom]{cancel}		% Allows the use of \cancelto{}{} to cross out equat
\usepackage{tikz}
\usepackage{titlesec}

\theoremstyle{definition} % hace que el cuerpo sea upright (no itálico)
\newtheorem{theorem}{Teorema}[section]
\newtheorem{lemma}[theorem]{Lema}
\newtheorem{proposition}[theorem]{Proposición}
\newtheorem{corollary}[theorem]{Corolario}
\newtheorem{definition}[theorem]{Definición}
\newtheorem{example}[theorem]{Ejemplo}
\newtheorem{remark}[theorem]{Observación}
\newtheorem{note}[theorem]{Nota}
\newcommand{\R}{\mathbb{R}}
\newcommand{\N}{\mathbb{N}}
\newcommand{\Q}{\mathbb{Q}}
\newcommand{\Z}{\mathbb{Z}}
%========== CHANGE TITLE FORMAT  ======================
%\titleformat{\section}
%{\bfseries}	% format
%{\thesection}	% label
%{0.3cm}		% separation between label and body
%{}		% code preceding title body
%[]		% code following title body


%========== HEADER AND FOOTER CONFIGURATION ======================
\pagestyle{fancy}
% HEADER[EVEN PAGES]{ODD PAGES}
% Left header
\lhead[
	\scriptsize{\MakeUppercase{\leftUpperHeader}} % Even pages
]
{
	\scriptsize{\MakeUppercase{\leftUpperHeader}} % Odd pages
}

% Central header
\chead[]{}

% Right header
\rhead[
\scriptsize{\rightUpperHeader} % Even pages
]
{
\scriptsize{\rightUpperHeader} % Odd pages
}

\renewcommand{\headrulewidth}{0.8pt}	% Width of the header rule

% FOOTER[EVEN PAGES]{ODD PAGES}
% Left footer
\lfoot[]{}

% Central footer
\cfoot[
\tiny{\pageCounterName\space\thepage\space \pageSeparator\space\pageref{LastPage}} % Even pages
]
{
\tiny{\pageCounterName\space\thepage\space \pageSeparator\space\pageref{LastPage}} % Odd pages
}

% Right footer
\rfoot[]{}
\renewcommand{\footrulewidth}{0.8pt} %Width of the footer rule


%================================== USER OWN FUNCTIONS ===============================================
\newcommand{\PDEA}[1]{\cdot 10^{#1}} % Función para escribir más rápidamente multiplicaciones por potencias de 10. PDEA = Por Diez Elevado A

\usepackage{xargs}	%Permite manejar argumentos opcionales en los comandos creados por el usuario

%Incluir imágenes, la sintaxis es la siguiente:
%\IncludeImage{ruta}[escalado][pie figura][etiqueta], los corchetes son opcionales
\newcommandx*{\IncludeImage}[4][2=1, 3=, 4=]{
	%#1 es la ruta a la imagen
	%#2 es el escalado
	%#3 es el pie de figura
	%#4 es la etiqueta
	
	\begin{figure}[h!]
		\centering
		\includegraphics[width=#2\linewidth]{#1}
		%Si se especifica pie de figura...
		\ifblank{#3}{}{
			\caption{#3}	
		}
		%Si se especifica etiqueta...
		\ifblank{#4}{}{
			\label{#4}	
		}
	\end{figure}
	\FloatBarrier
}



%================================ DOCUMENT BEGINNING  =============================================
\begin{document}
%***************** TITLE PAGE (DO NOT MODIFY) ******************************
\begin{titlepage}
	\newgeometry{top=2cm, bottom=3.5cm}

  %Logos Escuela y universidad
	\def\logoSize{0.2}
	\begin{figure}
  \hfill
  \includegraphics[width=\logoSize\linewidth]{Image/UC COLOR-01.png}
	\end{figure}

	%Espacio entre logos y título
	\vspace*{1.75cm}

	%Título con interlineado aumentado
	\begin{spacing}{2}
	\centering{
	\Huge{
		\textbf{\reportTile}
	}
	}
	\end{spacing}
	
	%Línea horizontal de la portada
	\hrule
	
	%Se baja al final de la hoja
	\vfill
	\begin{flushright}
	\LARGE{\writter}

	\vspace{2cm}
	
	\Large
	\subject\\
	\course\\
	\university
	
	\vspace{1cm}
	
	\today
	\end{flushright}
	
\end{titlepage}
	
\newgeometry{top=3cm, bottom=3cm}

%\tableofcontents
%\cleardoublepage

\section{Problema 1}
\label{Problema 1}

Sea \(V\) un espacio vectorial real.

\begin{itemize}
  \item Llamamos la complejificación de \(V\), que denotamos por \(V_{\mathbb C}\), al producto \(V\times V\). Un elemento de \(V_{\mathbb C}\) es un par ordenado \((u,v)\), donde \(u,v\in V\); denotamos a tal elemento por \(u+iv\).
  \item Definimos la suma en \(V_{\mathbb C}\) mediante la regla
  \[
    (u_1+iv_1)+(u_2+iv_2)=(u_1+u_2)+i(v_1+v_2)
  \]
  para todo \(u_1,u_2,v_1,v_2\in V\).
  \item La multiplicación por un escalar se define como
  \[
    (a+ib)(u+iv)=(au-bv)+i(av+bu)
  \]
  para todo \(a,b\in\mathbb{R}\) y todo \(u,v\in V\).
\end{itemize}
Demuestre que con las definiciones anteriores, \(V_{\mathbb C}\) es un espacio vectorial sobre complejo.

\begin{proof}
  
\end{proof}

\section*{Problema 2}
Suponga que \(U\) es un subespacio vectorial de \(V\) con \(U\neq V\). Suponga además que \(S\in\mathcal L(U,W)\), para algún espacio vectorial \(W\), y que \(S\neq 0\) (es decir, suponga que \(Su\neq 0\) para algún \(u\in U\)). Defina \(T:V\to W\) mediante
\[
  Tv=
  \begin{cases}
    Sv, & \text{si } v\in U,\\[4pt]
    0,  & \text{si } v\in V \text{ y } v\not\in U.
  \end{cases}
\]
Determine si \(T\) es una transformación lineal.

\section*{Problema 3}
Considere el espacio vectorial \(\mathbb{P}_3(\R)\) de los polinomios de grado menor o igual a \(3\), y la función \(T:\mathbb{P}_3(\R)\to \mathbb{P}_3(\R)\) definida por
\[
  T(p(x)) = p(0)\,x^3 + p(1)\,(x-4)^2.
\]

\begin{enumerate}
  \item[(a)] Muestre que \(T\) es una transformación lineal.
    \begin{proof}
      Sean $p(x),q(x) \in \mathbb{P}_3(\R)$ 
        \begin{align*}
          T(p(x)+q(x))&=(q+p)(0)x^3+(p+q)(1)(x-4)^2 \\
          &=p(0)x^3 +q(0)x^3+p(1)(x-4)^2+q(1)(x-4)^2 \\
          &=p(0)x^3 +q(1)(x-4)^2+q(0)x^3+q(1)(x-4)^2 \\
          &=T(p(x))+T(q(x))
        \end{align*}
      Así, obtenemos que $T(p(x)+q(x))=T(p(x))+T(p(x)) \quad \forall p(x),p(x)\in \mathbb{P}_3(\R)$. Ahora, sea $\alpha \in \mathbb{F}$, tenemos que 
        \begin{align*}
          T(\alpha \cdot p(x)) &= (\alpha \cdot p(0)x^3)+(\alpha \cdot p(1)(x-4)^2) \\
          & = \alpha \cdotp(0)x^3 + \alpha \cdot p(1)(x-4)^2 \\
          & = \alpha(p(0)x^3 + q(1)(x-4)^2) \\
          & = \alpha \cdot T(p(x))
        \end{align*}
      Por lo tanto, $T$ es una transformación lineal. 
    \end{proof}
  \item[(b)] Encuentre la matriz representante de \(T\) respecto a la base canónica \(\{1,x,x^2,x^3\}\).
    \begin{proof}
      En primer lugar, calculamos los respectivos componentes de la base, para $T(1)$, necesitamos que $p(x)=1$, es decir $p(0)=1$ y $p(1)=1$, por lo tanto
      \begin{align*}
        T(1) &= p(0)x^3 + p(1)(x-4)^2 \\
        & = x^3 +(x-4)^2 \\ 
        & = 16- 8x+x^2 +x^3 \\
        & = 16 \cdot 1 - 8 \cdot x + 1\cdot x^2 + 1 \cdot x^3
      \end{align*}
      Ahora, notamos que para los siguientes tres vectores de la base, $T(x^1),T(x^2)$ y  $T(x^3)$, necesitamos que $p(x^1)=x^1, p(x^2)=x^2$ y $p(x^3)=x^3$, es decir, $p(0)=0$ y $p(1)=1$, por lo tanto
      \begin{align*}
        T(x) &= p(0)x^3+  p(1)(x-4)^2 \\
        & = 0 \cdot x^3 +1 \cdot (x-4)^2 \\
        & = 16-8 x+x^2 \\
        & = 16 \cdot 1 -8 \cdot x + 1\cdot x^2 + 0\cdot x~3
      \end{align*} 
      Finalmente, la matriz asociada a la transformación lineal respecto a $\{1,x,x^2,x^3\}$ viene determinada por los escalares de las respectivas combinaciones lineales, es decir:
      \[
      A = \begin{pmatrix}
        16 & 16 & 16 & 16 \\
        -8 & -8 & -8 & -8 \\
        1 & 1 & 1 & 1 \\
        1 & 0 & 0 & 0
      \end{pmatrix}
      \]
    \end{proof}
\end{enumerate}

\section*{Problema 4}
Considere los vectores en \(\mathbb R^2\)
\[
  v_1=\begin{pmatrix}4\\[2pt]3\end{pmatrix},\qquad
  v_2=\begin{pmatrix}1\\[2pt]4\end{pmatrix}.
\]
Sea \(P(v_1,v_2)\) el paralelogramo generado por ambos vectores, y sea \(A=[v_1,v_2]\) la matriz cuyas columnas son \(v_1\) y \(v_2\). Muestre que
\[
  \det A = \operatorname{Area}\big(P(v_1,v_2)\big).
\]
\begin{proof}
  Observamos como se ve el paralelogramo formado por $v_1,v_2 \in \R^2$.
    \begin{center}
    \begin{tikzpicture}[scale=0.75] % Ajustar escala para que se parezca al tamaño de la imagen
      % --- Configuración general de estilo ---
      % Flechas para los ejes (ajustado para tener flechas)
      \tikzstyle{axis} = [thick, ->, >=stealth'] 
      \tikzstyle{parallelogram_line} = [thin] % Líneas del paralelogramo más finas
      
      % --- Grilla (como en la imagen, líneas finas y sin color) ---
      \draw[gray!50, very thin, step=1] (0,0) grid (5,7);

      % --- Ejes Coordenados (con flechas al final) ---
      \draw[axis] (-0.5,0) -- (5.5,0) node[below right] {$x$};
      \draw[axis] (0,-0.5) -- (0,7.5) node[left] {$y$}; % 'node[left]' para la 'y' al final

      % --- Coordenadas de los vértices ---
      \coordinate (O) at (0,0);
      \coordinate (v1_coords) at (4,3);
      \coordinate (v2_coords) at (1,4);
      \coordinate (sum_coords) at (5,7); % v1 + v2

      % --- Dibujar el Paralelogramo (solo contorno, sin relleno como en la imagen) ---
      % Lado v1 desde el origen
      \draw[parallelogram_line] (O) -- (v1_coords);
      % Lado v2 desde el origen
      \draw[parallelogram_line] (O) -- (v2_coords);
      % Lado paralelo a v1 desde v2
      \draw[parallelogram_line] (v2_coords) -- (sum_coords);
      % Lado paralelo a v2 desde v1
      \draw[parallelogram_line] (v1_coords) -- (sum_coords);

      % --- Etiquetar los vértices ---
      \node[below left] at (O) {$(0,0)$};
      \node[above right] at (v1_coords) {$(4,3)$};
      % Ajuste aquí: 'above right' para moverlo del eje y
      \node[above right] at (v2_coords) {$(1,4)$}; 
      \node[above right] at (sum_coords) {$(5,7)$};

    \end{tikzpicture}
  \end{center}
  Ahora, consideramos dos áreas axuliares para calcular área$(P(v_1,v_2))$

  \begin{figure}[h!] 
\centering % Centra el conjunto de las dos imágenes en la página

% --- INICIO DE LA PRIMERA FIGURA (CON ETIQUETAS FINALES AJUSTADAS) ---
\begin{minipage}{0.48\textwidth}
    \centering
    \begin{tikzpicture}[scale=0.7]
        % --- Estilos ---
        \tikzstyle{axis} = [thick, ->, >=stealth']
        \tikzstyle{parallelogram_line} = [thin]

        % --- Grilla ---
        \draw[gray!50, very thin, step=1] (0,0) grid (5,7);

        % --- Ejes Coordenados ---
        \draw[axis] (-0.5,0) -- (5.5,0) node[below right] {$x$};
        \draw[axis] (0,-0.5) -- (0,7.5) node[left] {$y$};

        % --- Coordenadas ---
        \coordinate (O) at (0,0);
        \coordinate (v1_coords) at (4,3);
        \coordinate (v2_coords) at (1,4);
        \coordinate (sum_coords) at (5,7);

        % --- Rectángulo contenedor ---
        \draw[black, thick, dashed] (0,0) rectangle (5,7);

        % --- Etiquetas del Rectángulo (AJUSTE FINAL) ---
        \node[below left, xshift=-2pt, yshift=-2pt] at (0,0) {$(0,0)$};
        \node[below, yshift=-2pt] at (5,0) {$(5,0)$}; % <-- CAMBIO: Posición 'below' para no tapar la 'x'
        \node[above right] at (0,7) {$(0,7)$}; % <-- CAMBIO: Posición 'above right' para no tapar la 'y'
        \node[above right] at (5,7) {$(5,7)$};

        % --- Paralelogramo ---
        \draw[parallelogram_line, black, thick] (O) -- (v1_coords) -- (sum_coords) -- (v2_coords) -- cycle;
    \end{tikzpicture}
\end{minipage}%
\hfill % Espacio flexible entre las figuras
% --- INICIO DE LA SEGUNDA FIGURA (CON ETIQUETAS CENTRADAS Y SIN COORDENADAS) ---
\begin{minipage}{0.48\textwidth}
    \centering
    \begin{tikzpicture}[scale=0.7]
        % --- Estilos ---
        \tikzstyle{axis} = [thick, ->, >=stealth']
        \tikzstyle{parallelogram_line} = [thin]

        % --- Grilla ---
        \draw[gray!50, very thin, step=1] (0,0) grid (5,7);

        % --- Ejes Coordenados ---
        \draw[axis] (-0.5,0) -- (5.5,0) node[below right] {$x$};
        \draw[axis] (0,-0.5) -- (0,7.5) node[left] {$y$};

        % --- Coordenadas ---
        \coordinate (O) at (0,0);
        \coordinate (v1_coords) at (4,3);
        \coordinate (v2_coords) at (1,4);
        \coordinate (sum_coords) at (5,7);

        % --- Áreas exteriores (con grises suaves) ---
        \fill[gray!10] (O) -- (0,4) -- (v2_coords) -- cycle;
        \fill[gray!15] (O) -- (4,0) -- (v1_coords) -- cycle;
        \fill[gray!20] (v2_coords) -- (1,7) -- (sum_coords) -- cycle;
        \fill[gray!25] (v1_coords) -- (5,3) -- (sum_coords) -- cycle;

        % --- Nuevos rectángulos ---
        \fill[gray!30] (0,4) rectangle (1,7); \draw[black] (0,4) rectangle (1,7);
        \fill[gray!35] (4,0) rectangle (5,3); \draw[black] (4,0) rectangle (5,3);

        % --- Contenedor y Paralelogramo ---
        \draw[black, thick] (0,0) rectangle (5,7);
        \draw[parallelogram_line, black, thick] (O) -- (v1_coords) -- (sum_coords) -- (v2_coords) -- cycle;

        % --- ETIQUETAS DE ÁREA PERFECTAMENTE CENTRADAS ---
        \node at (0.33, 2.67) {A1};
        \node at (2.67, 1.0)  {A2};
        \node at (2.33, 6.0)  {A3};
        \node at (4.67, 4.33) {A4};
        \node at (0.5, 5.5)   {A5};
        \node at (4.5, 1.5)   {A6};
        
    \end{tikzpicture}
\end{minipage}

\end{figure}
 
  Notemos que podemos calcular ambas áreas, ya que la primera corresponde a un rectangulo y la segunda es la suma de rectángulos y triángulos rectángulos. Por lo tanto, la resta de áreas es igual a
  \begin{align*}
    A_1-A_2 & = 35 -22 \\
    & = 13
  \end{align*}
  Además, podemos calular $\det (A)$:
  \begin{align*}
    \det (A) &= \det \begin{vmatrix}
      4 & 1 \\
      3 & 4
    \end{vmatrix} \\
    & = 4\cdot 4 - 1 \cdot 3 \\
    & = 13
  \end{align*}
  Con esto, concluimos que \det (A)= Área$(P(v_1,v_2))$.
\end{proof}

\section*{Problema 5}
Sea \(A\in M_{n\times n}(\R)\). Decimos que \(A\) es \emph{nilpotente} de orden \(k\) si \(A^k=0\) para algún \(k\in\Z_{>0}\). Por otro lado, decimos que \(A\) es \emph{ortogonal} si \(A^{T}A=I\). Pruebe las siguientes propiedades:
\begin{enumerate}
  \item[(a)] El determinante de toda matriz nilpotente es \(0\).
    \begin{proof}
      Sea $A \in M_{n\times n}(\R)$ una matriz nilpotente, es decir, para algún $k \in \Z > 0$ se tiene que $A^k=0$. En particular, para dicho caso $k$ se tiene que 
      \[
          M_0 =   \begin{vmatrix}
          0 & 0 & 0 & \cdots & 0\\[4pt]
          0 & 0 & 0 & \cdots & 0\\[4pt]
          \vdots & \vdots & \vdots & \ddots & \vdots\\[4pt]
          0 & 0 & 0 & \cdots & 0
          \end{vmatrix} = \det(A^k)=\det (A\cdot A \cdots A)
      \]
      Luego, sabemos que para $A,B \in M_{n\times n}(R)$, entonces 
      \[
        \det (AB)=\det (A) \det (B)
      \]
      tenemos que 
      \[
        \det (A\cdot A\cdots \A)= \det (A) \cdot (A) \cdots \det (A) 
      \]
      \[
        M_0 = \det(A) \cdot \det (A) \cdots \det (A)
      \]  
      Por lo tanto, 
      \[
        \det (A) = 0
      \]
    \end{proof}
  \item[(b)] El determinante de toda matriz ortogonal es \(\pm 1\).
  \begin{proof}
    Sea $A\in \M_{n\times n}(\R)$ una matriz ortogonal. Definimos $L = \det (A)$ y utilizamos
    \[
      \det (A)= \det (A^t) \text{ y } \det(AB)=\det (A) \det (B) \quad \forall A,B\in M_{x \times n}(\R) 
    \]
    tenemos que $L = \det (A)= \det (A^t)$. Por lo tanto
    \begin{align*}
      1 &= \det (A) \\
      & = \det (A^t A) \\
      &= \det (A^t)\det (A)\\
      &= L \times L \\
      &= L^2
    \end{align*}
    Entonces
    \[
      L=\det(A)=\pm 1
    \]
  \end{proof}
\end{enumerate}

\section*{Problema 6}
Sea \(A\) una matriz cuadrada. Muestre que las matrices triangulares por bloque
\[
  \begin{pmatrix} I & \ast\\[2pt] 0 & A\end{pmatrix},\qquad
  \begin{pmatrix} A & \ast\\[2pt] 0 & I\end{pmatrix},\qquad
  \begin{pmatrix} I & 0\\[2pt] \ast & A\end{pmatrix},\qquad
  \begin{pmatrix} A & 0\\[2pt] \ast & I\end{pmatrix},
\]
tienen todas ellas determinante igual a \(\det A\). (En la notación anterior, el símbolo ``\(\ast\)'' significa ``lo que sea'').

\section*{Problema 7}
El objetivo de este problema es demostrar el determinante de Vandermonde
\[
  \begin{vmatrix}
    1 & c_0 & c_0^2 & \cdots & c_0^n\\[4pt]
    1 & c_1 & c_1^2 & \cdots & c_1^n\\[4pt]
    \vdots & \vdots & \vdots & \ddots & \vdots\\[4pt]
    1 & c_n & c_n^2 & \cdots & c_n^n
  \end{vmatrix}
  = \prod_{0\le j<k\le n} (c_k-c_j).
\]

Para ello, procederemos por inducción:
\begin{enumerate}
  \item[(a)] Muestre que la fórmula funciona para \(n=1,2\).
  \item[(b)] Defina \(x:=c_n\) y muestre que el determinante es un polinomio de grado \(n\),
  \[
    A_0 + A_1 x + A_2 x^2 + \cdots + A_n x^n,
  \]
  donde los coeficientes dependen de \(c_0,c_1,\dots,c_{n-1}\).
  \item[(c)] Muestre que las raíces del polinomio anterior son \(x=c_0,c_1,\dots,c_{n-1}\), de manera que el determinante toma la forma
  \[
    A_n (x-c_0)(x-c_1)\cdots(x-c_{n-1}).
  \]
  \item[(d)] Asumiendo que la fórmula es válida para \(n-1\), calcule \(A_n\) y demuestre la fórmula para \(n\).
\end{enumerate}

 
\end{document}
